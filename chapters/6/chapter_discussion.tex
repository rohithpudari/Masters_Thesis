\startchapter{Discussion, Limitations \& Implications}
\label{chapter:discussion}

\section{Introduction}
We began this thesis with an analysis of Copilot code suggestions on language idioms and code smells, to understand the current capabilities of \cct{} like Copilot.
In this chapter, we try to address \textbf{RQ-2} (Given the current boundary, how far is it from suggesting design decisions?) with a discussion on the complex nature of design decisions involving many factors, making it difficult for \cct{} like Copilot to detect the information from code files and suggest design decisions to satisfy the top abstraction level of our taxonomy.

Next, in section~\ref{cs2design} we discuss our vision for \cct{} like Copilot to satisfy design level in our taxonomy, and outline the difficulty its underlying Codex LLM approach might run into.
Further, in section~\ref{evolution} we discuss how design choices change over time and outline the difficulties of \cct{} like Copilot to keep updating its suggestions and reflect the current design practices.
Furthermore, having established the software abstraction hierarchy to help assess the capabilities of \cct{}, in section~\ref{limitations} we report on the threats to the validity of the research presented in this thesis and the steps we have taken to mitigate these threats.

We conclude this chapter by discussing some of the ways in which this study could be extended further, some implications for researchers and practitioners, and some future works that our study enables~(section~\ref{future}).

\section{\cct{} to design level}
\label{cs2design}
% \subsection{Design Patterns}
% \label{patterns}
To satisfy design level, \cct{} should be capable of capturing design and module level concerns. 
These include capturing design patterns~(such as Observer) and architectural tactics~(such as Heartbeat) to improve and personalize suggestions.
To do this, however, the training data must also use these patterns and best practices.
Currently, Copilot does not support multi-file input, So it is not possible to evaluate its design suggestions, as software development process may include multiple folders with a file structure. 

\cct{} should be able to adapt their suggestions to context specific issues such as variable naming conventions and formatting. 
This would be challenging as the existing guidelines are not standard in this space and mostly depend on context.
The vision for a \cct{} like Copilot is to be capable of suggesting patterns like Model-View-Controller (MVC) when prompted.

\subsection{Evolution of design over time}
\label{evolution}
Design practices evolve over time. \cct{} need to update their suggestions at regular intervals to reflect the changes in design practices. 
For example, in JavaScript, callback api was considered as the best practice in the past to achieve concurrency which were replaced by promises. 
While testing, Copilot suggested code which is specifically mentioned in the JavaScript documentation as a common bad practice and anti-pattern\footref{docs}.

Bad Practices in using promises for asynchronous JavaScript like not returning promises after creation, forgetting to terminate chains without catch statement, which are explained in documentation\footnote{\label{docs}\url{https://developer.mozilla.org/en-US/docs/Web/JavaScript/Guide/Using_promises}} and StackOverflow\footnote{\url{https://stackoverflow.com/questions/30362733/handling-errors-in-promise-all/}} are not known to Copilot and suggested code with those common anti-patterns as they could have occurred more frequently in Copilot training data. 
However, this is beyond the scope of this study and will be part of future work.
\section{Limitations}

\section{Ethical Considerations}
\section{Explainability}
\label{explain}
\cop{} is closed source, and it is currently not possible to determine the source or the reason behind each suggestion, making it difficult to detect any problems (access is only via an API). 
However, engineering software systems is laden with ethical challenges, and understanding why a suggestion was made, particularly for architectural questions such as system fairness, is essential. 
Probes, as introduced in \cite{karmakar21}, might expand technical insight into the models.

Another challenge is understanding the basis for the ranking metric for different suggestions made by Copilot. This metric has not been made public. Thus, we cannot determine the reason Copilot is ranking one approach (e.g., non-idiomatic) over the idiomatic (preferred) approach. However, large language model suggestions are based on its training data~\cite{training_extraction}, so one explanation is that the non-idiomatic approach is more frequent in the training data~\cite{stochastic_parrots}. Better characterization of the rankings would allow users to better understand the motivation. 
\section{Control}
\label{control}
Being generative models, tools like Copilot are extremely sensitive to input with stability challenges and to make them autonomous raises control concerns.
For example, if a human asks for a N\textsuperscript{2} sorting algorithm, should Copilot recommend one, or the NlogN alternative? 
Ideally, tools should warn users if prompted to suggest sub-optimal code. 
\AISE{} should learn to differentiate between optimal and sub-optimal code. 
One direction to look at is following commit histories of files, as they are the possible places to find bug fixes and performance improvements.

\section{Implications}
% Practical implications can be separated into two categories, pre-training and at suggestion time. 
\subsection{Implications for practice}
For \emph{pre-training the LLM} (e.g., Codex), \AISE{} tools will need higher-quality training data. This might be addressed by carefully engineering training examples and filtering out known flaws, code smells, and bad practices. Careful data curation seems to be part of the approach already~\cite{alphacode}. However, there is little clarity on how this process happens and how to evaluate suggestions, particularly for non-experts. One approach is to add more verified sources like well known books and code documentation pages to follow the best practices. 
Pre-training might rank repositories for training input according to code quality (e.g., only repositories with acceptable coding standards). %The goal here is to make good practices more frequent than the bad ones to make Copilot suggest good code as its top suggestion.

For \emph{code completion time}, \AISE{} tools could collaborate with, or be used in conjunction with, existing tools for code smells like SonarQube\footnote{https://www.sonarqube.org} or other code review bots, to potentially improve the quality of suggestions. Since developers are increasingly expecting to wait for a code suggestion, the returned results could be filtered for quality. Active learning approaches which learn a user's context (e.g., the company coding style) would also improve suggestion acceptability. %Many of these insights can be taken from existing knowledge in the information retrieval domain. 

% As code completion tools are used for productivity, they should improve the ranking system to make best solutions appear as top suggestion (it would take more time to go through all suggestions)
% Recommendations to improve \AIDE{}:
% \begin{enumerate}
%     \item add more verified sources like books and documentations to training data.
%     \item perform code smell detection and vulnerability checks before every suggestion and rank them accordingly.
%     \item ranking suggestions based on repository (source of suggestion) popularity.
% \end{enumerate}

\subsection{Implications for researchers}
\subsubsection{Moving Beyond Tokens}
Another research challenge is to move beyond token-level suggestions and work at code block or file level (e.g., a method or module). 
Increasing the model input size to span multiple files and folders would improve suggestions. For example, when there are multiple files implementing the MVC pattern, Copilot should never suggest code where \textsf{Model} communicates directly with \textsf{View}. 
\AISE{} tools will need to make suggestions in multiple program units to accommodate these more abstract design concerns.

One suggestion is to use recent ML advances in helping language models `reason', such as the chain of thought process by Wang et al.~\cite{chain_of_thought}. 
Chain-of-thought shows the model an example of reasoning, which then allows the model to reproduce the reasoning pattern on a different input.
Such reasoning is common for design questions. 
Shokri~\cite{shokri21} explored this with framework sketches.

For example, using architectural scenarios helps (humans) reason about which tactic is most suitable for the scenario~\cite{kazman98}. This is a version of chain of thought for designing systems. 
However, we have an imperfect understanding of the strategies that drive human design approaches for software~\cite{Arab2022}. 
\section{Future Directions}
\label{future}
To solve complex software engineering challenges with AI, machine learning models must be able to capture design and module level concerns. 
These include recapturing design patterns (such as Observer) and architectural tactics (such as Heartbeat) to improve and personalize suggestions. 
To do this, however, the training data must also use these patterns and best practices.
The vision is for something like Copilot to be capable of suggesting patterns like Model-View-Controller (MVC) when prompted. 
Being able to identify where design artifacts occur in the code, such as with pattern recovery~\cite{Keim2020}, is one avenue to explore. 
\section{Chapter Summary}