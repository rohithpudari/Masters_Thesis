\startchapter{Discussion, Limitations \& Implications}
\label{chapter:discussion}

\section{Introduction}
We began this thesis with an analysis of Copilot code suggestions on language idioms and code smells, to understand the current capabilities of \cct{} like Copilot.
In this chapter, we try to address \textbf{RQ-2} (Given the current boundary, how far is it from suggesting design decisions?) with a discussion on the complex nature of design decisions involving many factors, making it difficult for \cct{} like Copilot to detect the information from code files and suggest design decisions to satisfy the top abstraction level of our taxonomy.

Next, in section~\ref{cs2design} we discuss our vision for \cct{} like Copilot to satisfy design level in our taxonomy, and outline the difficulty its underlying Codex LLM approach might run into.
Further, in section~\ref{evolution} we discuss how design choices change over time and outline the difficulties of \cct{} like Copilot to keep updating its suggestions and reflect the current design practices.
Furthermore, having established the software abstraction hierarchy to help assess the capabilities of \cct{}, in section~\ref{limitations} we report on the threats to the validity of the research presented in this thesis and the steps we have taken to mitigate these threats.

We conclude this chapter by discussing some of the ways in which this study could be extended further, some implications for researchers and practitioners, and some future works that our study enables~(section~\ref{future}).

\section{Code Smells to Design level}
\label{cs2design}


\subsection{Design Patterns}
\label{patterns}
To satisfy design level, \cct{} should be capable of capturing design and module level concerns. 
These include recapturing design patterns~(such as Observer) and architectural tactics~(such as Heartbeat) to improve and personalize suggestions.
To do this, however, the training data must also use these patterns and best practices.
The vision for a \cct{} like Copilot is to be capable of suggesting patterns like Model-View-Controller (MVC) when prompted.

\subsection{Evolution of design over time}
\label{evolution}
Design practices evolve over time. \cct{} need to update their suggestions at regular intervals to reflect the changes in design practices. 

For example, in JavaScript, callback api was considered as the best practice in the past to achieve concurrency which were replaced by promises. Copilot suggested code which is specifically mentioned in the JavaScript documentation as a bad practice and anti-pattern\footref{docs}.

Bad Practices in using promises for asynchronous JavaScript like not returning promises after creation, forgetting to terminate chains without catch statement, which are explained in documentation\footnote{\label{docs}\url{https://developer.mozilla.org/en-US/docs/Web/JavaScript/Guide/Using_promises}} and StackOverflow\footnote{\url{https://stackoverflow.com/questions/30362733/handling-errors-in-promise-all/}} are not known to Copilot and suggested code with those common anti-patterns as they could have occurred more frequently in Copilot training data. However, this is beyond the scope of this study and will be part of future work.
\section{Limitations}

\section{Ethical Considerations}
\section{Explainability}
\label{explain}
Copilot is closed source, and it is currently not possible to determine the source or the reason behind each suggestion, making it difficult to detect any problems (access is only via an API). 
However, engineering software systems is laden with ethical challenges, and understanding why a suggestion was made, particularly for architectural questions such as system fairness, is essential. 
Probes, as introduced in \cite{karmakar21}, might expand technical insight into the models.

Another challenge is understanding the basis for the ranking metric for different suggestions made by Copilot. This metric has not been made public. Thus, we cannot determine the reason Copilot is ranking one approach (e.g., non-idiomatic) over the idiomatic (preferred) approach. However, large language model suggestions are based on its training data~\cite{training_extraction}, so one explanation is that the non-idiomatic approach is more frequent in the training data~\cite{stochastic_parrots}. Better characterization of the rankings would allow users to better understand the motivation. 
\subsubsection{Control}
\label{control}
Being generative models, tools like Copilot are extremely sensitive to input with stability challenges, and to make them autonomous raises control concerns.
For example, if a human asks for a N\textsuperscript{2} sorting algorithm, should Copilot recommend one or the NlogN alternative? 
Ideally, tools should warn users if prompted to suggest sub-optimal code. 
\AISE{} should learn to differentiate between optimal and sub-optimal code. 
One direction to look at is following commit histories of files, as they are the possible places to find bug fixes and performance improvements.

\section{Implications}

\subsection{Research implications}

\subsection{Practical implications}


\section{Future Directions}
\label{future}
To solve complex software engineering challenges with AI, machine learning models must be able to capture design and module level concerns. 
These include recapturing design patterns (such as Observer) and architectural tactics (such as Heartbeat) to improve and personalize suggestions. 
To do this, however, the training data must also use these patterns and best practices.
The vision is for something like Copilot to be capable of suggesting patterns like Model-View-Controller (MVC) when prompted. 
Being able to identify where design artifacts occur in the code, such as with pattern recovery~\cite{Keim2020}, is one avenue to explore. 

\subsection{Moving Beyond Tokens}
Another research challenge is to move beyond token-level suggestions and work at code block or file level (e.g., a method or module). 
Increasing the model input size to span multiple files and folders would improve suggestions. For example, when there are multiple files implementing the MVC pattern, Copilot should never suggest code where \textsf{Model} communicates directly with \textsf{View}. 
\AISE{} tools will need to make suggestions in multiple program units to accommodate these more abstract design concerns.

One suggestion is to use recent ML advances in helping language models `reason', such as the chain of thought process by Wang et al.~\cite{chain_of_thought}. 
Chain-of-thought shows the model an example of reasoning, which then allows the model to reproduce the reasoning pattern on a different input.
Such reasoning is common for design questions. 
Shokri~\cite{shokri21} explored this with framework sketches.

For example, using architectural scenarios helps (humans) reason about which tactic is most suitable for the scenario~\cite{kazman98}. This is a version of chain of thought for designing systems. 
However, we have an imperfect understanding of the strategies that drive human design approaches for software~\cite{Arab2022}. 
\section{Chapter Summary}
In summary, we start this chapter by showing the methodology used in addressing textbf{RQ-1} (What are the current boundaries of \cct{}?). 
We first introduced Pythonic idioms and best practices in JavaScript.
We then present our sampling approach for sampling 25 coding scenarios to analyze Copilot code suggestions.
Furthermore, we discussed the input given to Copilot to trigger a code suggestion 
and how the input was restricted to deriving the desired way from the input.
Finally, we described our evaluation approach for Copilot code suggestions.

We sampled 25 Pythonic idioms from Alexandru et al.~\cite{Alexandru2018}, and Farook et al.~\cite{idioms}.
We identified that Copilot did not suggest the idiomatic way as its top suggestion for 23 out of 25 coding scenarios in Python, which addressed \textbf{RQ-1.1} (How do \cct{} manage programming idioms?).
Furthermore, we sampled 25 best practices in JavaScript from the AirBNB JavaScript coding style guide~\cite{airbnb_code}. We identified that Copilot did not suggest the recommended best practice for 22 out of 25 coding scenarios in JavaScript, which addressed \textbf{RQ-1.2} (How do \cct{} manage to manage to suggest non-smelly code?).


% we showed that Copilot struggles to detect and most common idiomatic ways present in public repositories of GitHub and rank them higher than the non-idiomatic ways. The ideal behavior of \cct{} like Copilot in solving this problem is detecting common patterns present in code and rank them higher as the idiomatic ways for a task.
% In the next chapter (chapter~\ref{smells}), we look into how this ideal behavior can cause problems in the case of code smells, where common bad practices present in public repositories of GitHub can make \cct{} like Copilot introduce bad coding practices in its suggestions.

% % \section{Chapter Summary}
% In summary, we start this chapter by showing the methodology used in addressing \textbf{RQ-1.2} (How do \cct{} manage to suggest non-smelly code?). We first introduced the study setup with the input to Copilot and how it was restricted to deriving the best practice from the input and how the suggestions from Copilot were evaluated. We sampled best practices from AirBNB JavaScript coding style guide~\cite{airbnb_code}, and then compared it against Copilot suggestions. Based on results shown in Table~\ref{tab:all_bp}, Copilot struggles to suggest the best practices from widely used coding standards in its suggestions. 

In this chapter, we showed that Copilot struggles to detect and follow coding style guides present in public repositories of GitHub and always suggests code that follows those coding style guides. We also observed that Copilot struggles to detect and most common idiomatic ways present in public repositories of GitHub and rank them higher than the non-idiomatic ways. 
Identifying this delineation could help in urn AI-supported code completion tools such as Copilot into full-fledged AI-supported software engineering tools.

In the next chapter (chapter~\ref{chapter:framework}), we illustrate our taxonomy inspired by autonomous driving levels on the software abstraction hierarchy in \AISE{} and use the results shown in this chapter to delineate where \cct{} like Copilot currently stands in the taxonomy. 