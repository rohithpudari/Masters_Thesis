\startfirstchapter{Introduction}
\label{chapter:introduction}
Programming is a powerful and ubiquitous problem-solving tool. Developing systems that can assist software developers or even generate programs independently could make programming more productive and accessible~\cite{peggyprod}.
With increasing pressure on software developers to produce code quickly, there is considerable interest in tools and techniques for improving productivity~\cite{productivity}.
Code completion is one such feature that predicts what a software developer tries to code and offers predictions as suggestions to the user~\cite{cct}. All modern IDEs feature intelligent code completion tools in different forms that are used by both new and experienced software developers~\cite{cct_usage}. Developing AI systems that can effectively model and understand code can transform these code completion tools and how we interact with them~\cite{cct_usage}.

Recent large-scale pre-trained language models such as Codex~\cite{copilot} have demonstrated an impressive ability to generate code and can now solve programming contest-style problems~\cite{empirical_eval}. 
However, software development is much more than writing code. It involves complex challenges like following the best practices, avoiding code smells, using design patterns, and many more decisions before writing code.

The scope of capabilities for \cct{} is uncertain. Identifying the nature of Copilot capabilities when it comes to more complex challenges, i.e., \AISE{} (as opposed to programming tasks, such as coding or solving competitive programming problems). Delineating where \cct{} are currently best able to perform, and where more complex software development tasks overwhelm them helps answer questions like exactly which software problems can current \cct{} solve? 
If \cct{} make a suggestion, is that suggestion accurate and optimal? Should a user intervene to correct it? However, identifying these boundaries is a challenging task. In the next section, we discuss this challenge and the research opportunity it creates as this study's motivation.

\section{Motivation}
In recent years, there have been considerable improvements in the field of \cct{}. 
Copilot~\cite{Copilot-web}, an in-IDE recommender system that leverages OpenAI's Codex neural language model (NLM)~\cite{copilot} which uses a GPT-3 model~\cite{Gpt3} has been at the forefront and is particularly impressive in understanding the context and semantics of code with just a few lines of comments or code as input and can suggest the next few lines or even entire functions in some cases~\cite{copilot}.

The biggest challenge with using tools like Copilot is their training data. 
These tools are trained on existing software source code, and training costs are expensive.
Several classes of errors have been discovered (shown in section~\ref{challenges}), which follow from the presence of these same errors in public (training) data. 
There is a need to stay cautious about common bugs people make, creeping into Copilot suggestions and keeping Copilot updated to the ever-changing best practices or new bug fixes in public repositories.

Understanding the current limitations of Copilot will help developers use \cct{} effectively. 
The advantage of knowing where Copilot is good/bad lets users use these tools more efficiently by letting \cct{} take over where they excel and focus more on tasks where \cct{} are shown to have struggled with the quality of code suggestions.

A taxonomy of software abstractions would help create a formal structure to better understand the capabilities and limitations of \cct{} like Copilot. 
A taxonomy helps describe the hierarchical stages of software abstractions at which \cct{} like Copilot are operating. 
For example, Copilot is capable of suggesting syntactical code but cannot suggest architectural tactics.
Creating a taxonomy of software abstractions can be useful in developing new \cct{} and measuring their capabilities.

Another advantage of a taxonomy of software abstractions makes code suggestions better in complex situations, shifting research focus to make \cct{} better at tasks shown to have struggled with the quality of code suggestions, minimizing the input required by the user to make \cct{} create meaningful, quality suggestions.

In this thesis, we investigate the current limitations of Copilot with an empirical study on Copilot suggestions. Using the findings, we introduce a taxonomy of software abstraction hierarchy, modeled on the ideas in the SAE taxonomy and Koopman's extension for self-driving vehicles. The insights gained from this study can help developers understand how to best use tools like Copilot and provide insights to researchers trying to develop and improve the field of \cct{}.

\section{\cct{}}
In-IDE code completion tools have improved a lot in recent years, from suggesting variables or method calls from user code bases~\cite{mandelin2005} to suggesting entire code blocks~\cite{Ciniselli2021}. The deep learning resurgence has led to strong advances in the field of \cct{}. Pre-trained large language models such as GPT-3~\cite{Gpt3}, CodeBERT~\cite{codebert}, Codex~\cite{copilot}, AlphaCode~\cite{alphacode} have dramatically improved the state-of-the-art on a code completion tasks.

Introduced in June 2021, GitHub's Copilot~\cite{Copilot-web} is an in-IDE recommender system that leverages OpenAI's Codex large language model (LLM)~\cite{copilot} which uses a GPT-3 model~\cite{Gpt3} and is then fine-tuned on code from GitHub to generate code suggestions that are uncannily effective and are able to perform above human average on programming contest problems~\cite{empirical_eval}. Copilot can generate code in a variety of languages given some context such as comments, function names, and surrounding code. As GitHub Copilot is the largest and most capable such model currently available, we perform all our experiments on GitHub Copilot as a representative for \cct{}.
\section{Problem Statement and Research Questions}
The overarching goal of this study is to:
\begin{quote}
    Identify the boundaries of the capabilities of GitHub Copilot. Toward this goal, we compare Copilot code suggestions against well known language idioms and code smells. Additionally, we introduce a simple taxonomy of software abstraction hierarchies to show different capability levels of \cct{}. 
\end{quote}

The capabilities and the limitations of current \cct{} like Copilot are unknown, identifying the limitations of \cct{} would help the users use the tool effectively and focus more on the tasks \cct{} are shown to be not useful. 
The objective of this study is to achieve a better understanding of the areas Copilot performs better than a human and the areas where Copilot performs worse than a human. We conduct an exploratory study with the following research objectives:

\begin{enumerate}
  \item[\textbf{RQ-1: }]
  \textbf{What are the current boundaries of \cct{}?} \\
  \textbf{Approach -} We use GitHub's Copilot as a representative for \cct{}. We explore Copilot's code suggestions for code smells and usage of language idioms. We conduct additional investigation to determine the current boundaries of Copilot by introducing a taxonomy of software abstraction hierarchies where ‘basic programming functionality’ such as code compilation and syntax checking is at the least abstract level. Software architecture analysis and design is at the most abstract level. 
  
  \item[\textbf{RQ-1.1: }]
  \textbf{How do \cct{} manage programming Idioms?} \\
  \textbf{Approach -} We examine Copilot code suggestions on top 25 idioms used in open source projects sampled from work of Alexandru et al.~\cite{Alexandru2018}, which identified idioms from presentations given by renowned Python developers. We investigate how Copilot's top code suggestion compares to Python idioms from Alexandru et al.~\cite{Alexandru2018}. In addition, we report if the idiom is listed in any of the 10 viewable suggestions from Copilot.
  
  \item[\textbf{RQ-1.2: }]
  \textbf{How do \cct{} manage manage to suggest non-smelly code?} \\
\textbf{Approach -} We examine Copilot code suggestions on 25 different best practices sampled from AirBNB JavaScript coding style guide~\cite{airbnb_code}. We investigate how Copilot's top code suggestion compares to the best practices in AirBNB JavaScript coding style guide~\cite{airbnb_code}. Additionally, we report if the best practice is listed in any of the 10 viewable suggestions from Copilot. 
 
  \item[\textbf{RQ-2: }]
  \textbf{Given the current boundary, how far is it from suggesting design decisions which seem much beyond the boundary??} \\
  \textbf{Approach -} Based on our findings in RQ-1, we discuss how far current \cct{} are from the design level in our taxonomy. We look at current limitations of Copilot and provide recommendations on how to make current \cct{} reach design abstraction level. Additionally, we report on ethical considerations, explainability and control of \cct{} like Copilot. 
\end{enumerate}
\section{Contributions}
\section{Thesis Outline}


