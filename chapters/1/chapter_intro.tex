\startfirstchapter{Introduction}
\label{chapter:introduction}
Programming is a powerful and ubiquitous problem-solving tool. Developing systems that can assist software developers or even generate programs independently could make programming more productive and accessible. 
Code completion is a feature that predicts what a software developer is trying to code and offers predictions as suggestions to the user. All modern IDEs feature intelligent code completion tools in different forms and it is used by both new and experienced software developers. Developing AI systems that can effectively model and understand code can transform these code completion tools and the way we interact with them.

Recent large-scale pre-trained language models such as \cop{}~\cite{Copilot-web} have demonstrated an impressive ability to generate code and are now able to solve programming contest style problems~\cite{empirical_eval}. 
However, software engineering is much more than writing code, it involves complex challenges like choosing the best practices, avoiding code smells, using design patterns and many more decisions before writing code. 
The scope of capabilities for \cct{} is uncertain. Identifying the nature of \cop{} capabilities when it comes to more complex challenges, i.e., \AISE{} (as opposed to development tasks, such as coding or programming problems). Delineating where \cct{} are currently best able to perform, and where more complex software engineering tasks overwhelm them is helpful in answering questions like 
Exactly which software problems can current \cct{} solve? 
If \cct{} makes a suggestion, is that suggestion accurate and optimal? Should a user intervene to correct it? But identifying these boundaries is a challenging task. In the next section, we discuss this challenge and the research opportunity that it creates as the motivation for this study.

\section{Motivation}


In this thesis, we investigate the

\section{\cct{}}




\section{Problem Statement and Research Questions}
The overarching goal of this study is to:
\begin{quote}
    Identify the boundaries of the capabilities of GitHub Copilot. Toward this goal, we compare Copilot code suggestions against well known language idioms and code smells. Additionally, we introduce a simple taxonomy of software abstraction hierarchies to show different capability levels of \cct{}. 
\end{quote}

The capabilities and the limitations of current \cct{} like Copilot are unknown, identifying the limitations of \cct{} would help the users use the tool effectively and focus more on the tasks \cct{} are shown to be not useful. 
The objective of this study is to achieve a better understanding of the areas Copilot performs better than a human and the areas where Copilot performs worse than a human. We conduct an exploratory study with the following research objectives:

\begin{enumerate}
  \item[\textbf{RQ-1: }]
  \textbf{What are the current boundaries of \cct{}?} \\
  \textbf{Approach -} We use GitHub's Copilot as a representative for \cct{}. We explore Copilot's code suggestions for code smells and usage of language idioms. We conduct additional investigation to determine the current boundaries of Copilot by introducing a taxonomy of software abstraction hierarchies where ‘basic programming functionality’ such as code compilation and syntax checking is at the least abstract level. Software architecture analysis and design is at the most abstract level. 
  
  \item[\textbf{RQ-1.1: }]
  \textbf{How do \cct{} manage programming Idioms?} \\
  \textbf{Approach -} We examine Copilot code suggestions on top 25 idioms used in open source projects sampled from work of Alexandru et al.~\cite{Alexandru2018}, which identified idioms from presentations given by renowned Python developers. We investigate how Copilot's top code suggestion compares to Python idioms from Alexandru et al.~\cite{Alexandru2018}. In addition, we report if the idiom is listed in any of the 10 viewable suggestions from Copilot.
  
  \item[\textbf{RQ-1.2: }]
  \textbf{How do \cct{} manage manage to suggest non-smelly code?} \\
\textbf{Approach -} We examine Copilot code suggestions on 25 different best practices sampled from AirBNB JavaScript coding style guide~\cite{airbnb_code}. We investigate how Copilot's top code suggestion compares to the best practices in AirBNB JavaScript coding style guide~\cite{airbnb_code}. Additionally, we report if the best practice is listed in any of the 10 viewable suggestions from Copilot. 
 
  \item[\textbf{RQ-2: }]
  \textbf{Given the current boundary, how far is it from suggesting design decisions which seem much beyond the boundary??} \\
  \textbf{Approach -} Based on our findings in RQ-1, we discuss how far current \cct{} are from the design level in our taxonomy. We look at current limitations of Copilot and provide recommendations on how to make current \cct{} reach design abstraction level. Additionally, we report on ethical considerations, explainability and control of \cct{} like Copilot. 
\end{enumerate}
\section{Contributions}
\section{Thesis Outline}


