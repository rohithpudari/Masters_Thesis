\startfirstchapter{Introduction}
\label{chapter:introduction}
Programming is a powerful and ubiquitous problem-solving tool. Developing systems that can assist software developers or even generate programs independently could make programming more productive and accessible~\cite{peggyprod}.
With increasing pressure on software developers to produce code quickly, there is considerable interest in tools and techniques for improving productivity~\cite{productivity}.
Code completion is one such feature that predicts what a software developer is trying to code and offers predictions as suggestions to the user~\cite{cct}. All modern IDEs feature intelligent code completion tools in different forms and it is used by both new and experienced software developers~\cite{cct_usage}. Developing AI systems that can effectively model and understand code can transform these code completion tools and the way we interact with them~\cite{cct_usage}.

Recent large-scale pre-trained language models such as Codex~\cite{copilot} have demonstrated an impressive ability to generate code and are now able to solve programming contest style problems~\cite{empirical_eval}. 
However, software engineering is much more than writing code. It involves complex challenges like following the best practices, avoiding code smells, using design patterns and many more decisions before writing code. 

The scope of capabilities for \cct{} is uncertain. Identifying the nature of Copilot capabilities when it comes to more complex challenges, i.e., \AISE{} (as opposed to development tasks, such as coding or programming problems). Delineating where \cct{} are currently best able to perform, and where more complex software engineering tasks overwhelm them is helpful in answering questions like 
exactly which software problems can current \cct{} solve? 
If \cct{} make a suggestion, is that suggestion accurate and optimal? Should a user intervene to correct it? However, identifying these boundaries is a challenging task. In the next section, we discuss this challenge and the research opportunity that it creates as the motivation for this study.

\section{Motivation}
There have been huge improvements in the field of \cct{} in recent years. Copilot has been at the forefront and is particularly impressive in understanding the context and semantics of code with just a few lines of comments or code as input and can suggest the next few lines or even entire functions in some cases~\cite{copilot}.

The biggest challenge with using tools like Copilot is their training data, These tools are trained on existing software source code, and training costs are expensive.
Several classes of errors have been discovered (shown in section~\ref{challenges}), which follow from the presence of these same errors in public (training) data. There is a need to stay cautious about common bugs that people make creeping into Copilot suggestions and to keep Copilot updated to the ever-changing best practices or new bug fixes in public repositories.

Understanding the current limitations of Copilot will help developers use \cct{} effectively. The advantage of knowing where Copilot is good/bad lets users use these tools more efficiently by letting \cct{} take over where they excel and focus more on tasks where \cct{} are shown to have struggled with the quality of code suggestions.

Another advantage of a taxonomy of software abstractions make code suggestions better in complex situations, shifting research focus to make \cct{} better at tasks shown to have struggled with the quality of code suggestions, minimizing the input required by the user to make \cct{} create meaningful, quality suggestions.

In this thesis, we investigate the current limitations of Copilot with an empirical study on Copilot suggestions. Using the findings, we introduce a taxonomy of software abstraction hierarchy, modeled on the ideas in the SAE taxonomy and Koopman's extension for self-driving vehicles. The insights gained from this study can help developers understand how to best use tools like Copilot as well as provide insights to researchers trying to develop and improve the field of \cct{}.

\section{\cct{}}
The purpose of code completion as an IDE feature is to save the user's time and effort by suggesting code before manually inputting it. Code completion can also aid in project exploration by giving new users a chance to view entities from various areas of the code base. The IDE typically offers a list of potential completions for the user because the process of developing code can vary greatly.

In-IDE code completion tools have improved a lot in recent years. Early code completion techniques include suggesting variables or method calls from user code bases~\cite{mandelin2005}, based on lists alphabetically sorted lists that show the next token to write given the registered characters.
Followed by tools capable of suggesting entire code blocks~\cite{Ciniselli2021} utilizing statistical language models, such as n-gram models, is the origin of the use of data-driven strategies for code recommendation. 

Karampatsis et al.~\cite{karampatsis} suggested that the most effective code-completion approach is based on neural network models. 
The authors used byte pair encoding~\cite{BPE} as a method of overcoming the lack of vocabulary issue, demonstrating that their best model performs better than n-gram models and is easily transferable to other domains resulting in a deep learning resurgence that has led to strong advances in the field of \cct{}. 

Pre-trained large language models such as GPT-3~\cite{Gpt3}, CodeBERT~\cite{codebert}, Codex~\cite{copilot}, AlphaCode~\cite{alphacode} have dramatically improved the state-of-the-art on a code completion. 
For example, even when the large language model's only input is a plain language description of the developer's desired implementation, these models are now capable of predicting the entire method with reasonable accuracy~\cite{copilot}.
\section{Problem Statement and Research Questions}

\section{Research Design and Methodology}
The current \cct{} approaches focus on programming-in-the-small~\cite{DeRemer1976} i.e, on individual lines of code. 
Code language models have focused (effectively!) on source code as natural language~\cite{natural}.
This models the software development task as predicting the next token or series of tokens.

Introduced in June 2021, GitHub's Copilot~\cite{Copilot-web} is an in-IDE recommender system that leverages OpenAI's Codex neural language model (NLM)~\cite{copilot} which uses a GPT-3 model~\cite{Gpt3} and is then fine-tuned on code from GitHub to generate code suggestions that are uncannily effective and can perform above human average on programming contest problems~\cite{empirical_eval}. As Copilot's webpage says, Copilot aims to produce ``safe and effective code [with] suggestions for whole lines or entire functions'' as ``your AI pair programmer''~\cite{Copilot-web}. 
Thus, support from language models is currently focused on software \textit{programming}, rather than software \emph{development} (in-the-large).
Copilot can generate code in various languages given some context, such as comments, function names, and surrounding code. Copilot is the largest and most capable model currently available. We perform all our experiments on GitHub Copilot as a representative for \cct{}.
% GitHub's Copilot is one such \cct{} that can generate code in various languages
% given some context such as comments, function names, and surrounding code. As
% GitHub Copilot is the largest and most capable such model currently available. We base all our findings on GitHub's Copilot as a representative of \cct{} for this study.

We begin by sampling the top 25 language idioms used in open source projects from Alexandru et al.~\cite{Alexandru2018} and Farook et al.~\cite{idioms}. We then compare and contrast Copilot's code suggestion for each idiom and report if Copilot suggested the idiom listed in Alexandru et al.~\cite{Alexandru2018} and Farook et al.~\cite{idioms}.
We also sampled 25 coding scenarios for detecting code smells from AirBNB JavaScript coding style guide~\cite{airbnb_code}, a widely used coding style and code review standard. We then compare and contrast Copilot's code suggestion for each scenario and report if Copilot suggested the best practice listed in the AirBNB JavaScript coding style guide. 
We base our taxonomy of software abstraction hierarchies on the findings from Copilot suggestions on language idioms and code smells.

We then report on the current limitations of \cct{} using Copilot as a representative tool and introduce a taxonomy of software abstraction hierarchies (see figure~\ref{fig:taxonomy}), inspired by an analogous concept in the more developed (but still nascent) area of autonomous driving. 
We also present an example coding scenario for every level of abstraction in our taxonomy, showing the requirements to be fulfilled by \cct{} to satisfy that level of abstraction.

We conclude by providing a discussion for future development of \cct{} to reach the design level of abstraction in our taxonomy and discuss the limitations of our work. 

\section{Contributions}

This thesis contributes the following:

\begin{itemize}
    \item We demonstrate the current limitations of Copilot's code suggestions. We show that Copilot does not perform well in suggesting best practices and language idioms. We present our discussion on challenges by using \cct{} like Copilot.
    \item We release our coding experiments on Copilot for best practices and language idioms. We make this available in our \repl{}.
    \item Using the findings from Copilot's code suggestions on best practices and language idioms, we present a taxonomy is a software abstraction hierarchy where ‘basic programming functionality’ such as code compilation and syntax checking is at the least abstract level, including the set of requirements for \cct{} to satisfy each level of abstraction.
    \item Based on our experiences in this study, we present future directions for moving beyond code completion to AI-supported software engineering, which will require an AI system that can, among other things, understand how to avoid code smells, follow language idioms, and eventually propose rational software designs.
\end{itemize}
\section{Thesis Outline}

This thesis is organized as follows:

\begin{description}
\item[Chapter 2] elaborates the background information and some related work on \cct{} and GitHub Copilot. It further introduces the challenges with using \cct{} that will be explored in this thesis.
\item[Chapter 3] introduces our study and the methodology showing the sampling approach, input, and evaluation criteria we used to address \textbf{RQ-1} (What are the current boundaries of code completion tools).
We then present the results of Copilot code suggestions for language idioms and code smells.
\item[Chapter 4] introduces a taxonomy of software abstraction hierarchy, inspired by SAE autonomous driving safety levels. It then presents a set of requirements for \cct{} to satisfy each level of abstraction.
\item[Chapter 5] addresses \textbf{RQ-2} (Given the current boundary, how far is it from suggesting design decisions?) with a discussion on the complex nature of design decisions. In addition, we discuss future directions for \cct{} to reach the design abstraction level in the taxonomy. 
We conclude by discussing the implications and the limitations of this study.
\item[Chapter 6] presents the conclusion of this research study.
\end{description}