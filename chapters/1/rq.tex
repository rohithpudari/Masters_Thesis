\section{Problem Statement and Research Questions}
The overarching goal of this study is to:
\begin{quote}
    Identify the boundaries of the capabilities of GitHub Copilot. We compare Copilot code suggestions against well-known Pythonic idioms and JavaScript code smells toward this goal. Additionally, we introduce a simple taxonomy of software abstraction hierarchies to show different capability levels of \cct{}. 
\end{quote}

The capabilities and the limitations of current \cct{} like Copilot are unknown. Identifying the boundaries of \cct{} would help the users use the tool effectively and shift the research focus more on the tasks \cct{} are proven to be not helpful. 

This study aims to understand the areas where Copilot performs better than a human and where Copilot performs worse than a human. 
We conduct an exploratory study with the following research objectives:

\begin{enumerate}
  \item[\textbf{RQ-1: }]
  \textbf{What are the current boundaries of \cct{}?} \\
  \textbf{Approach -} We use GitHub's Copilot as a representative for current \cct{}. We explore Copilot's code suggestions for code smells and usage of Pythonic idioms. We conduct additional investigation to determine the current boundaries of Copilot by introducing a taxonomy of software abstraction hierarchies where ‘basic programming functionality’ such as code compilation and syntax checking is at the least abstract level. Software architecture analysis and design are at the most abstract level. 
  
  \item[\textbf{RQ-1.1: }]
  \textbf{How do \cct{} manage programming idioms?} \\
  \textbf{Approach -} We investigate Copilot code suggestions on the top 25 Pythonic idioms used in open source projects. These Pythonic idioms are sampled from the work of Alexandru et al.~\cite{Alexandru2018} and Farook et al.~\cite{idioms}, which identified Pythonic idioms from presentations given by renowned Python developers. We investigate how Copilot's top code suggestion compares to Python idioms from Alexandru et al.~\cite{Alexandru2018} and Farook et al.~\cite{idioms}. In addition, we report if the Pythonic idiom is listed in any of the ten viewable suggestions from Copilot.
  
  \item[\textbf{RQ-1.2: }]
  \textbf{How do \cct{} manage manage to suggest non-smelly code?} \\
\textbf{Approach -} We investigate Copilot code suggestions on 25 different best practices in JavaScript. We sampled best practices from the AirBNB JavaScript coding style guide~\cite{airbnb_code}. We explore how Copilot's top code suggestion compares to the best practices recommended in the AirBNB JavaScript coding style guide~\cite{airbnb_code}. Additionally, we report if the best practice is listed in any of the ten viewable suggestions from Copilot. 
 
  \item[\textbf{RQ-2: }]
  \textbf{Given the current boundary, how far is it from suggesting design decisions which seem much beyond the boundary?} \\
  \textbf{Approach -} Based on our findings in RQ-1, we discuss how far current \cct{} are from the design level in our taxonomy. We look at the current limitations of Copilot and provide recommendations on how to make current \cct{} reach the design abstraction level. Additionally, we report on ethical considerations, explainability, and control of \cct{} like Copilot. 
\end{enumerate}