\section{Problem Statement and Research Questions}
The overarching goal of this study is to:
\begin{quote}
    Identify the boundaries of the capabilities of GitHub Copilot. Toward this goal, we explore how often Copilot follows best practices and avoids code smells. Additionally, we introduce a simple taxonomy of software abstractions to show the capabilities of current \cct{}. 
\end{quote}

As discussed in the previous sections, The capabilities of current \cct{} like Copilot are unknown, identifying the limitations of \cct{} would help the users use the tool effectively and focus more on the tasks \cct{} are shown to be not useful. The objective of this study is to achieve a better understanding of the areas Copilot performs better than a human and the areas where Copilot performs worse than a human. We conduct an exploratory studey with the following research objectives:

\begin{enumerate}
  \item[\textbf{RQ-1: }]
  \textbf{What are the current boundaries of \cct{}?} \\
  \textbf{Approach -} this is the approach area
  
  \item[\textbf{RQ-1.1: }]
  \textbf{How do \cct{} manage manage to write non-smelly code?} \\
  \textbf{Approach -} this is the approach area
  
  \item[\textbf{RQ-1.2: }]
  \textbf{How do \cct{} manage programming Idioms?} \\
  \textbf{Approach -} this is the approach area
  
  \item[\textbf{RQ-2: }]
  \textbf{Given the current boundary, how far is it from suggesting design decisions which seem much beyond the boundary??} \\
  \textbf{Approach -} this is the approach area
\end{enumerate}