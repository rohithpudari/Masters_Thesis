\section{Thesis Outline}

This thesis is organized as follows:

\begin{description}
\item[Chapter 2] elaborates the background information and some related work on \cct{} and GitHub Copilot. It further introduces the challenges with using \cct{} that would be explored in this thesis.
\item[Chapter 3] introduces our study and the methodology showing the sampling approach, input, and evaluation criteria we used to address \textbf{RQ-1} (What are the current boundaries of code completion tools).
we then present the results of Copilot code suggestions for language idioms and code smells.
\item[Chapter 4] introduces taxonomy of software abstraction hierarchy, inspired from SAE autonomous driving safety levels. It then presents set of requirements for \cct{} to satisfy each level of abstraction.
\item[Chapter 5] addresses \textbf{RQ-2} (Given the current boundary, how far is it from suggesting design decisions?) with a discussion on the complex nature of design decisions. In addition, we provide a discussion on future directions for \cct{} to reach design abstraction level in the taxonomy. 
We conclude by discussing the implications and the limitations of this study.
\item[Chapter 6] presents the conclusion of this research study.
\end{description}


