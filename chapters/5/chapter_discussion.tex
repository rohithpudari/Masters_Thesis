\startchapter{Discussion, Limitations \& Implications}
\label{chapter:discussion}

\section{Introduction}
We began this thesis by analyzing Copilot code suggestions on Pythonic idioms, and Javascript code smells to understand the current boundaries of \cct{} like Copilot using a software abstraction taxonomy.
In this chapter, we try to address \textbf{RQ-2} (Given the current boundary, how far is it from suggesting design decisions?) with a discussion on the complex nature of design decisions involving factors ranging from requirement analysis to maintenance, making it difficult for \cct{} like Copilot to detect the information from code files and suggest design decisions to satisfy the top software abstraction level of our taxonomy.

In section~\ref{cs2design}, we discuss our vision for \cct{} like Copilot to satisfy the design level in our taxonomy and outline the difficulty its underlying Codex LLM approach might run into.
Further, we discuss how design choices change over time and outline the difficulties of \cct{} like Copilot to keep updating its suggestions and reflect the current design practices~(section~\ref{evolution}).
Having established the software abstraction hierarchy to help assess the capabilities of \cct{}, in section~\ref{limitations}, we report on the threats to the validity of the research presented in this thesis and the steps we have taken to mitigate these threats.

We conclude this chapter by discussing how this study could be extended further, some implications for researchers and practitioners, and some future works that our study enables~(section~\ref{future}).

\section{Code Smells to Design level}
\label{cs2design}

\section{Implications}
\label{implications}
% Practical implications can be separated into two categories, pre-training and at suggestion time. 
This research helps guide future \cct{} to support software development. 
Good \cct{} has many potential uses, from recommending expanded code completions to optimizing code blocks. Automating code production could increase the productivity of current programmers. 

Future code generation models may enable developers to work at a higher degree of abstraction that hides specifics, similar to how contemporary software engineers no longer frequently write in assembly.
Good \cct{} may improve accessibility to programming or aid in training new programmers. Models could make suggestions for different, more effective, or idiomatic methods to implement programs, enabling one to develop their coding style.

\subsection{Implications for practitioners}

\subsubsection{Pre-training the LLM}
For pre-training the LLM~(e.g., Codex), 
\AISE{} tools will need higher-quality training data. This might be addressed by carefully engineering training examples and filtering out known flaws, code smells, and bad practices. Careful data curation seems to be part of the approach already~\cite{alphacode}. However, there is little clarity on how this process happens and how to evaluate suggestions, particularly for non-experts. One approach is to add more verified sources like well-known books and code documentation pages to follow the best practices. 
Pre-training might rank repositories for training input according to code quality (e.g., only repositories with acceptable coding standards).
%The goal here is to make good practices more frequent than the bad ones to make Copilot suggest good code as its top suggestion.

\subsubsection{code completion time}
% For \emph{code completion time}, 
\AISE{} tools could collaborate with, or be used in conjunction with, existing tools for code smells like SonarQube\footnote{https://www.sonarqube.org} or other code review bots to potentially improve the quality of suggestions. Since developers expect to wait for a code suggestion, the results could be filtered for quality. 
Amazon's code completion tool `CodeWhisperer' comes with a `run security scan' option, which performs a security scan on the project or file that is currently active in VS Code~\cite{amazon}.
Active learning approaches which learn a user's context (e.g., the company coding style) would also improve suggestion acceptability. %Many of these insights can be taken from existing knowledge in the information retrieval domain. 

% As code completion tools are used for productivity, they should improve the ranking system to make best solutions appear as top suggestion (it would take more time to go through all suggestions)
% Recommendations to improve \AIDE{}:
% \begin{enumerate}
%     \item add more verified sources like books and documentations to training data.
%     \item perform code smell detection and vulnerability checks before every suggestion and rank them accordingly.
%     \item ranking suggestions based on repository (source of suggestion) popularity.
% \end{enumerate}

\subsubsection{Over-reliance}
Over-reliance on generated outputs is one of the main hazards connected to the use of code generation models in practice.
Codex may provide solutions that seem reasonable on the surface but do not truly accomplish what the user had in mind. Depending on the situation, this could negatively impact inexperienced programmers and have serious safety consequences. Human oversight and vigilance are always required to safely use \cct{} like Copilot.
Empirical research is required to consistently ensure alertness in practice across various user experience levels, UI designs, and tasks.

\subsubsection{Ethical Considerations}
\label{ethics}
Packages or programs created by third parties are frequently imported within a code file. Software engineers rely on functions, libraries, and APIs for the majority of what called as ``boilerplate" code rather than constantly recreating the wheel. 
However, there are numerous choices for each task: For machine learning, use PyTorch or TensorFlow; for data visualization, use Matplotlib or Seaborn; etc.

Reliance on import suggestions from \cct{} like Copilot may increase as they get used to using \cct{}. 
Users may employ the model as a decision-making tool or search engine as they get more adept at "prompt engineering" with Codex. 
Instead of searching the Internet for information on "which machine learning package to employ" or "the advantages and disadvantages of PyTorch vs. Tensorflow," a user may now type "\# import machine learning package" and rely on Codex to handle the rest.
Based on trends in its training data, Codex imports substitutable packages at varying rates~\cite{copilot}, which may have various effects. 
Different import rates set by Codex may result in subtle mistakes when a particular import is not advised, increased robustness when an individual's alternative package would have been worse, and/or an increase in the dominance of an already powerful group of people and institutions in the software supply chain.

As a result, certain players may solidify their position in the package market, and Codex may be unaware of any new packages created following the first collection of training data. The model may also recommend deprecated techniques for packages that are already in use. Additional research is required to fully understand the effects of code creation capabilities and effective solutions.

% Despite the fact that many packages are free, developers and companies with popular packages can earn incentives, and free packages can act as covers for paid items. 
% Therefore, the import patterns used by Codex and other code generation models may have significant economic effects on individuals who create and maintain packages, in addition to possible safety or security effects.

\subsection{Implications for researchers}
With a wide range of applications, including programming accessibility, developer tools, and computer science education, effective code generation models have the potential to have a positive, revolutionary effect on society. 
However, like most technologies, these models may enable applications with societal drawbacks that we need to watch out for, and the desire to make a positive difference does not, in and of itself, serve as a defense against harm.
One challenge researchers should consider is that as capabilities improve, it may become increasingly difficult to guard against “automation bias.”

\subsubsection{Moving Beyond Tokens}
\label{tokens}
Another research challenge is to move beyond token-level suggestions and work at the code block or file level (e.g., a method or module). 
Increasing the model input size to span multiple files and folders would improve suggestions. For example, when there are multiple files implementing the MVC pattern, Copilot should never suggest code where \textsf{Model} communicates directly with \textsf{View}. 
\AISE{} tools will need to make suggestions in multiple program units to accommodate these more abstract design concerns.

One suggestion is to use recent ML advances in helping language models `reason', such as the chain of thought process by Wang et al.~\cite{chain_of_thought}. 
Chain-of-thought shows the model and example of reasoning, allowing the model to reproduce the reasoning pattern on a different input.
Such reasoning is common for design questions. 
Shokri~\cite{shokri21} explored this with framework sketches.

For example, using architectural scenarios helps (humans) reason about which tactic is most suitable for the scenario~\cite{kazman98}. This is a version of the chain of thought for designing systems. 
However, we have an imperfect understanding of the strategies that drive human design approaches for software~\cite{Arab2022}. 
\section{Threats to Validity}
\label{limitations}
Copilot and its underlying OpenAI Codex LLM are not open sources. 
%Accessing them is restricted to API calls, and General users cannot directly examine the model. 
We base our conclusions on API results, which complicate versioning and clarity. There are several threats to the validity of the work presented in this thesis. In this section, we summarize the dangers and also present the steps taken to mitigate them. We use the stable Copilot extension release (version: 1.30.6165) in Visual Studio Code. %The manual effort needed to query Copilot restricts the ability to use large datasets.

\subsection{Internal Validity}
Copilot is sensitive to user inputs, which hurts replicability as a different formulation of the problem might produce a different set of suggestions. 
Because Copilot uses Codex, a generative model, its outputs cannot be precisely duplicated. Copilot can produce various responses for the same request at . Copilot is a closed-source, black-box application that runs on a distant server and is therefore inaccessible to general users~(such as the author of this thesis).
Thus a reasonable concern is that our (human) input is unfair to Copilot, and with some different inputs, the tool might generate the correct idiom. 
For replicability, we archived all examples in our replication package at \repl{}.
%\neil{can we try this? }\rohith{idioms are mostly one-lined, and longer inputs/suggestions could be described as bad practices/design smells}

\subsection{Construct Validity}
The taxonomy of the software abstraction hierarchy presented in this thesis relies on our view of software abstractions.
Other approaches for classifying software abstractions~(such as the user's motivation for initiating \cct{}) might result in different taxonomy.
The hierarchy of software abstractions presented in this thesis relies on our understanding of software abstractions, and the results of Copilot code suggestions on language idioms and code smells. Further, we present our results using Python and JavaScript. It is possible that using some other programming language or \cct{} might have different results.

We intended to show where Copilot cannot consistently generate the preferred answer. We biased our evaluation to highlight this by choosing input that simulates what a less experienced programmer might enter. 
But we argue this is reasonable: for one, these are precisely the developers likely to use Copilot suggestions and unlikely to know the idiomatic usage.
More importantly, a lack of suggestion stability seems to come with its own set of challenges, which are equally important to understand.

\subsection{Bias, Fairness, and Representation}
Code generation models are susceptible to repeating the flaws and biases of their training data, just like natural language models~\cite{Gpt3}. These models can reinforce and maintain societal stereotypes when trained on a variety of corpora of human data, having a disproportionately negative effect on underprivileged communities. 
Additionally, bias may result in outdated APIs or low-quality code that reproduces problems, compromising performance and security. This might result in fewer people using new programming languages or libraries.

Codex has the ability to produce code that incorporates stereotypes regarding gender, ethnicity, emotion, class, name structure, and other traits~\cite{copilot}.
This issue could have serious safety implications, further motivating us to prevent over-reliance, especially in the context of users who might over-rely on Codex or utilise it without properly thinking through project design.

% (to use an analogy, we would expect all self-driving vehicles to make similar choices when confronted with similar circumstances). 
%this gets to the top levels of Koopmann's pyramid.
% Access to Copilot is currently restricted. 
% Working with GitHub's API - not open
%
%OpenAI and open source

\subsection{Ethical Considerations}
\label{ethics}
Packages or programs created by third parties are frequently imported within a code file. Software engineers rely on functions, libraries, and APIs for the majority of what called as ``boilerplate" code rather than constantly recreating the wheel. 
However, there are numerous choices for each task: For machine learning, use PyTorch or TensorFlow; for data visualization, use Matplotlib or Seaborn; etc.

Reliance on import suggestions from \cct{} like Copilot may increase as they get used to using \cct{}. 
Users may employ the model as a decision-making tool or search engine as they get more adept at "prompt engineering" with Codex. 
Instead of searching the Internet for information on "which machine learning package to employ" or "the advantages and disadvantages of PyTorch vs. Tensorflow," a user may now type "\# import machine learning package" and rely on Codex to handle the rest.
Based on trends in its training data, Codex imports substitutable packages at varying rates~\cite{copilot}, which may have various effects. 
Different import rates set by Codex may result in subtle mistakes when a particular import is not advised, increased robustness when an individual's alternative package would have been worse, and/or an increase in the dominance of an already powerful group of people and institutions in the software supply chain.

As a result, certain players may solidify their position in the package market, and Codex may be unaware of any new packages created following the first collection of training data. The model may also recommend deprecated techniques for packages that are already in use. Additional research is required to fully understand the effects of code creation capabilities and effective solutions.

% Despite the fact that many packages are free, developers and companies with popular packages can earn incentives, and free packages can act as covers for paid items. 
% Therefore, the import patterns used by Codex and other code generation models may have significant economic effects on individuals who create and maintain packages, in addition to possible safety or security effects.

\section{Explainability}
\label{explain}
Copilot is closed source, and it is currently not possible to determine the source or the reason behind each suggestion, making it difficult to detect any problems (access is only via an API). 
However, engineering software systems is laden with ethical challenges, and understanding why a suggestion was made, particularly for architectural questions such as system fairness, is essential. 
Probes, as introduced in \cite{karmakar21}, might expand technical insight into the models.

Another challenge is understanding the basis for the ranking metric for different suggestions made by Copilot. This metric has not been made public. Thus, we cannot determine the reason Copilot is ranking one approach (e.g., non-idiomatic) over the idiomatic (preferred) approach. However, large language model suggestions are based on its training data~\cite{training_extraction}, so one explanation is that the non-idiomatic approach is more frequent in the training data~\cite{stochastic_parrots}. Better characterization of the rankings would allow users to better understand the motivation. 
\subsubsection{Control}
\label{control}
Being generative models, tools like Copilot are extremely sensitive to input with stability challenges, and to make them autonomous raises control concerns.
For example, if a human asks for a N\textsuperscript{2} sorting algorithm, should Copilot recommend one or the NlogN alternative? 
Ideally, tools should warn users if prompted to suggest sub-optimal code. 
\AISE{} should learn to differentiate between optimal and sub-optimal code. 
One direction to look at is following commit histories of files, as they are the possible places to find bug fixes and performance improvements.

\section{Future Directions}
\label{future}
To solve complex software engineering challenges with AI, machine learning models must be able to capture design and module level concerns. 
These include recapturing design patterns (such as Observer) and architectural tactics (such as Heartbeat) to improve and personalize suggestions. 
To do this, however, the training data must also use these patterns and best practices.
The vision is for something like Copilot to be capable of suggesting patterns like Model-View-Controller (MVC) when prompted. 
Being able to identify where design artifacts occur in the code, such as with pattern recovery~\cite{Keim2020}, is one avenue to explore. 
\section{Chapter Summary}
In this chapter, we addressed \textbf{RQ-2} (Given the current boundary, how far is it from suggesting design decisions?). We began with a discussion on current capabilities of Copilot and the desired capabilities of \cct{} at design abstraction level. We then observed that design evolves over time and \cct{} need to be updated often.

Furthermore, we provided implications of our study. Additionally, we recommend moving beyond token-level suggestions and using reasoning methods to make \cct{} reach the highest level of abstraction in our taxonomy~(design level).
Finally, we discussed the threats to the validity of our research and the steps taken to address those threats. We concluded by presenting the future directions of our study. 