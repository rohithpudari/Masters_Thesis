\startchapter{Discussion, Future Work and Conclusion}
\label{chapter:discussion}

\section{Introduction}
We began this thesis with an analysis of Copilot code suggestions on Pythonic idioms, and Javascript best practices to understand the current boundaries of AI-supported code completion tools like Copilot using a software abstraction taxonomy. 
In this chapter, we first begin by extending this discussion by comparing Copilot performance on Pythonic idioms and Javascript best practices. In section~\ref{performance}, we discuss the differences in the performance and ranking of Copilot code suggestions on Pythoic idoms and JavaScript best practices. We also discuss how Copilot was able to suggest idiomatic code for some coding scenarios.

Furthermore, Having established the software abstraction hierarchy to help assess the capabilities of \cct{}, 
in section~\ref{recite}, we discuss what it means to recite code from training data of \cct{} like Copilot.
Additionally, we discuss how code recitation is an ideal behavior for \cct{} like Copilot to suggest idiomatic code but not for code smells.

In this second part of this thesis, we discussed our taxonomy of software abstractions and the challenges involved in creating \cct{} that are capable of satisfying the design level of our taxonomy. In section~\ref{implications}, we report on some implications for researchers and practitioners.
Finally, in section~\ref{limitations}, we report on the threats to the validity of the research presented in this thesis.

% We conclude this chapter by discussing how this study could be extended further, some implications for researchers and practitioners, and some future works that our study enables~(section~\ref{future}).


\section{Copilot code suggestions}
\label{performance}
% in the discussion, speculate on why there is variance. When I say “speculate” I mean offer some reasoning that follows a logical process, based on some evidence. WHy is “accessing properties” #1 but “array callback” is lower? What are the commonalities here? 
Copilot code suggestions for our coding scenarios in Pythonic idioms and JavaScript best practices~(shown in Chapter~\ref{chapter:methodology}) has three possible outcomes: Copilot suggesting the recommended approach as its top suggestion~(ideal behavior), Copilot having the recommended approach in its top 10 code suggestions but not the top suggestion~(less ideal) and Copilot does not have the recommended approach in any of its suggestions~(worst case).

Copilot recommending the idiomatic approach as its \emph{top suggestion} is the ideal behavior for all \cct{}.
This may require the recommended approach to be the most popular method to solve the programming task. For example, Copilot successfully suggested the idiomatic approach as its top suggestion for idioms 7~(set comprehension) and 10~(if condition check value). 
Both those programming tasks do not have more than 2 popular methods of solving the problem. Similarly, for code smells, Copilot successfully suggested the JavaScript best practice as its top suggestion for best practices 7~(accessing properties), 15~(converting array-like object to array), and 23~(check boolean value). These programming tasks also have the best practice approach as the most common approach used. 
To make \cct{} like Copilot suggest recommended approach as the top code suggestion, the approach must be the most common way of solving the given programming task. 

Copilot recommending the idiomatic approach in its \emph{top 10 suggestions} is not the ideal behavior but better than not having the idiomatic approach in its suggestions at all. 
Copilot had 8 idiomatic approaches and 5 best practices in its top 10 suggestions. This case is a result of \cct{} having the recommended approach in its training data, but the ranking metrics such as the popularity of the approach made the idiomatic approach rank below the non-idiomatic approach. 
To resolve this case, \cct{} like Copilot should update their ranking approach to have multiple metrics such as repository popularity or acceptance of the approach in online forums such as StackOverflow to make the idiomatic approach rank higher than the non-idiomatic approaches.

When the idiomatic approach is \emph{not in any of its top 10 suggestions}, \cct{} like Copilot does not have the recommended approach in its training data, making it unaware of the approach or the recommended approach is very rare that it didn't even make it to top 10 suggestions, this is the worst case of all three outcomes of our coding scenarios with Copilot. 
This also suggests that people do not use the recommended approach in their code, and efforts should be made to improve awareness of such recommended approaches to solve those common programming tasks.
\section{Copilot Code Recitation}
\label{recite}
% not here but in the discussion the thesis should discuss what it means to “recite” an idiom. Why or why not does Copilot not recommend a particular idiom? Readers will not understand the technical reasons - explain it.

\cct{} like Copilot are trained on billions of lines of code. The code suggestions Copilot makes to a user are adapted to the specific coding scenario, but the processing behind each code suggestion is ultimately taken from training~(public) code.
A recent study conducted by Bender et al.~\cite{stochastic_parrots} showed that LLMs like GPT-3~\cite{Gpt3} and Codex~\cite{copilot} could recite code suggestions identical to the code present in training data, which in turn can cause issues like license infringements~\cite{code_clone}.

Traditional N-gram LMs like SLANG~\cite{slang} and CACHECA~\cite{cacheca}~(shown in section~\ref{ngrams}) can only model relatively local dependencies, predicting each word given the preceding sequence of N words (usually 5 or fewer).
However, more advanced models like the Transformer LMs used in Codex~\cite{copilot} capture much larger windows and can produce code that is seemingly not only fluent but also coherent even over large code blocks. We were able to generate a whole 980 lines of a code file using Copilot with an input of the first 15 lines of that file\footnote{all experiments are documented in \repl{}}.  

This code recitation behavior of LLMs like Codex~\cite{copilot} can help with satisfying paradigms and idioms level of our taxonomy.
Idioms are the most accepted approaches in public~(training) data. 
The ideal behavior of \cct{} like Copilot is to recognize these idiomatic approaches to solve a programming task from training data and use them in code suggestions.

Similarly, Copilot can also recite common code smells in public~(training) data. \cct{} like Copilot needs to recognize the difference between an idiomatic approach and a code smell in training data. 
Metrics like code repository popularity and StackOverflow upvotes on the code can help \cct{} like Copilot to distinguish between idiomatic approaches and code smells. 
\section{Implications}
\label{implications}
% Practical implications can be separated into two categories, pre-training and at suggestion time. 
This research helps guide future \cct{} to support software development. 
Good \cct{} has many potential uses, from recommending expanded code completions to optimizing code blocks. Automating code production could increase the productivity of current programmers. 

Future code generation models may enable developers to work at a higher degree of abstraction that hides specifics, similar to how contemporary software engineers no longer frequently write in assembly.
Good \cct{} may improve accessibility to programming or aid in training new programmers. Models could make suggestions for different, more effective, or idiomatic methods to implement programs, enabling one to develop their coding style.

\subsection{Implications for practice}
For \emph{pre-training the LLM} (e.g., Codex), \AISE{} tools will need higher-quality training data. This might be addressed by carefully engineering training examples and filtering out known flaws, code smells, and bad practices. Careful data curation seems to be part of the approach already~\cite{alphacode}. However, there is little clarity on how this process happens and how to evaluate suggestions, particularly for non-experts. One approach is to add more verified sources like well-known books and code documentation pages to follow the best practices. 
Pre-training might rank repositories for training input according to code quality (e.g., only repositories with acceptable coding standards).
%The goal here is to make good practices more frequent than the bad ones to make Copilot suggest good code as its top suggestion.

For \emph{code completion time}, \AISE{} tools could collaborate with, or be used in conjunction with, existing tools for code smells like SonarQube\footnote{https://www.sonarqube.org} or other code review bots to potentially improve the quality of suggestions. Since developers expect to wait for a code suggestion, the results could be filtered for quality. 
Amazon's code completion tool `CodeWhisperer' comes with a `run security scan' option, which performs a security scan on the project or file that is currently active in VS Code~\cite{amazon}.
Active learning approaches which learn a user's context (e.g., the company coding style) would also improve suggestion acceptability. %Many of these insights can be taken from existing knowledge in the information retrieval domain. 

% As code completion tools are used for productivity, they should improve the ranking system to make best solutions appear as top suggestion (it would take more time to go through all suggestions)
% Recommendations to improve \AIDE{}:
% \begin{enumerate}
%     \item add more verified sources like books and documentations to training data.
%     \item perform code smell detection and vulnerability checks before every suggestion and rank them accordingly.
%     \item ranking suggestions based on repository (source of suggestion) popularity.
% \end{enumerate}


\subsection{Implications for researchers}
With a wide range of applications, including programming accessibility, developer tools, and computer science education, effective code generation models have the potential to have a positive, revolutionary effect on society. 
However, like most technologies, these models may enable applications with societal drawbacks that we need to watch out for, and the desire to make a positive difference does not, in and of itself, serve as a defense against harm.
One challenge researchers should consider is that as capabilities improve, it may become increasingly difficult to guard against “automation bias.”

\subsubsection{Moving Beyond Tokens}
\label{tokens}
Another research challenge is to move beyond token-level suggestions and work at the code block or file level (e.g., a method or module). 
Increasing the model input size to span multiple files and folders would improve suggestions. For example, when there are multiple files implementing the MVC pattern, Copilot should never suggest code where \textsf{Model} communicates directly with \textsf{View}. 
\AISE{} tools will need to make suggestions in multiple program units to accommodate these more abstract design concerns.

One suggestion is to use recent ML advances in helping language models `reason', such as the chain of thought process by Wang et al.~\cite{chain_of_thought}. 
Chain-of-thought shows the model and example of reasoning, allowing the model to reproduce the reasoning pattern on a different input.
Such reasoning is common for design questions. 
Shokri~\cite{shokri21} explored this with framework sketches.

For example, using architectural scenarios helps (humans) reason about which tactic is most suitable for the scenario~\cite{kazman98}. This is a version of the chain of thought for designing systems. 
However, we have an imperfect understanding of the strategies that drive human design approaches for software~\cite{Arab2022}. 
\section{Threats to Validity}
\label{limitations}
Copilot and its underlying OpenAI Codex LLM are not open sources. 
%Accessing them is restricted to API calls, and General users cannot directly examine the model. 
We base our conclusions on API results, which complicate versioning and clarity. There are several threats to the validity of the work presented in this thesis. In this section, we summarize the dangers and also present the steps taken to mitigate them. We use the stable Copilot extension release (version: 1.30.6165) in Visual Studio Code. %The manual effort needed to query Copilot restricts the ability to use large datasets.

\subsection{Internal Validity}
Copilot is sensitive to user inputs, which hurts replicability as a different formulation of the problem might produce a different set of suggestions. 
Because Copilot uses Codex, a generative model, its outputs cannot be precisely duplicated. Copilot can produce various responses for the same request at . Copilot is a closed-source, black-box application that runs on a distant server and is therefore inaccessible to general users~(such as the author of this thesis).
Thus a reasonable concern is that our (human) input is unfair to Copilot, and with some different inputs, the tool might generate the correct idiom. 
For replicability, we archived all examples in our replication package at \repl{}.
%\neil{can we try this? }\rohith{idioms are mostly one-lined, and longer inputs/suggestions could be described as bad practices/design smells}

\subsection{Construct Validity}
The taxonomy of the software abstraction hierarchy presented in this thesis relies on our view of software abstractions.
Other approaches for classifying software abstractions~(such as the user's motivation for initiating \cct{}) might result in different taxonomy.
The hierarchy of software abstractions presented in this thesis relies on our understanding of software abstractions, and the results of Copilot code suggestions on language idioms and code smells. Further, we present our results using Python and JavaScript. It is possible that using some other programming language or \cct{} might have different results.

We intended to show where Copilot cannot consistently generate the preferred answer. We biased our evaluation to highlight this by choosing input that simulates what a less experienced programmer might enter. 
But we argue this is reasonable: for one, these are precisely the developers likely to use Copilot suggestions and unlikely to know the idiomatic usage.
More importantly, a lack of suggestion stability seems to come with its own set of challenges, which are equally important to understand.

\subsection{Bias, Fairness, and Representation}
Code generation models are susceptible to repeating the flaws and biases of their training data, just like natural language models~\cite{Gpt3}. These models can reinforce and maintain societal stereotypes when trained on a variety of corpora of human data, having a disproportionately negative effect on underprivileged communities. 
Additionally, bias may result in outdated APIs or low-quality code that reproduces problems, compromising performance and security. This might result in fewer people using new programming languages or libraries.

Codex has the ability to produce code that incorporates stereotypes regarding gender, ethnicity, emotion, class, name structure, and other traits~\cite{copilot}.
This issue could have serious safety implications, further motivating us to prevent over-reliance, especially in the context of users who might over-rely on Codex or utilise it without properly thinking through project design.

% (to use an analogy, we would expect all self-driving vehicles to make similar choices when confronted with similar circumstances). 
%this gets to the top levels of Koopmann's pyramid.
% Access to Copilot is currently restricted. 
% Working with GitHub's API - not open
%
%OpenAI and open source

\subsection{Ethical Considerations}
\label{ethics}
Packages or programs created by third parties are frequently imported within a code file. Software engineers rely on functions, libraries, and APIs for the majority of what called as ``boilerplate" code rather than constantly recreating the wheel. 
However, there are numerous choices for each task: For machine learning, use PyTorch or TensorFlow; for data visualization, use Matplotlib or Seaborn; etc.

Reliance on import suggestions from \cct{} like Copilot may increase as they get used to using \cct{}. 
Users may employ the model as a decision-making tool or search engine as they get more adept at "prompt engineering" with Codex. 
Instead of searching the Internet for information on "which machine learning package to employ" or "the advantages and disadvantages of PyTorch vs. Tensorflow," a user may now type "\# import machine learning package" and rely on Codex to handle the rest.
Based on trends in its training data, Codex imports substitutable packages at varying rates~\cite{copilot}, which may have various effects. 
Different import rates set by Codex may result in subtle mistakes when a particular import is not advised, increased robustness when an individual's alternative package would have been worse, and/or an increase in the dominance of an already powerful group of people and institutions in the software supply chain.

As a result, certain players may solidify their position in the package market, and Codex may be unaware of any new packages created following the first collection of training data. The model may also recommend deprecated techniques for packages that are already in use. Additional research is required to fully understand the effects of code creation capabilities and effective solutions.

% Despite the fact that many packages are free, developers and companies with popular packages can earn incentives, and free packages can act as covers for paid items. 
% Therefore, the import patterns used by Codex and other code generation models may have significant economic effects on individuals who create and maintain packages, in addition to possible safety or security effects.

\section{Explainability}
\label{explain}
Copilot is closed source, and it is currently not possible to determine the source or the reason behind each suggestion, making it difficult to detect any problems (access is only via an API). 
However, engineering software systems are laden with ethical challenges, and understanding why a suggestion was made, particularly for architectural questions such as system fairness, is essential. 
Probes, as introduced in \cite{karmakar21}, might expand technical insight into the models.

Another challenge is understanding the basis for the ranking metric for different suggestions made by Copilot. 
This metric has not been made public. 
Thus, we cannot determine why Copilot ranks one approach (e.g., non-idiomatic) over the idiomatic (preferred) approach. However, large language model code suggestions are based on its training data~\cite{training_extraction}, so one explanation is that the non-idiomatic approach is more frequent in the training data~\cite{stochastic_parrots}. 
Better characterization of the rankings would allow users to better understand the motivation. 
\section{Control}
\label{control}
Being generative models, tools like Copilot are extremely sensitive to input with stability challenges and to make them autonomous raises control concerns.
For example, if a human asks for a N\textsuperscript{2} sorting algorithm, should Copilot recommend one, or the NlogN alternative? 
Ideally, tools should warn users if prompted to suggest sub-optimal code. 
\AISE{} should learn to differentiate between optimal and sub-optimal code. 
One direction to look at is following commit histories of files, as they are the possible places to find bug fixes and performance improvements.

\section{Future Directions}
\label{future}
To solve complex software engineering challenges with AI, machine learning models must be able to capture design and module level concerns. 
These include recapturing design patterns (such as Observer) and architectural tactics (such as Heartbeat) to improve and personalize suggestions. 
To do this, however, the training data must also use these patterns and best practices.
The vision is for something like Copilot to be capable of suggesting patterns like Model-View-Controller (MVC) when prompted. 
Being able to identify where design artifacts occur in the code, such as with pattern recovery~\cite{Keim2020}, is one avenue to explore. 
\section{Conclusion}
\label{conclusion}

% Furthermore, we provided implications for our study. Additionally, we recommend moving beyond token-level suggestions and using reasoning methods to make \cct{} reach the highest level of software abstraction in our taxonomy~(design level).
% Finally, we discussed the threats to the validity of our research and the steps taken to address those threats. We concluded by presenting the future directions of our study. 

Chapter~\ref{chapter:methodology} has shown the current challenges of \cct{} like security issues and license infringements. 
We also showed that \cct{} like Copilot struggle to use Pythonic idioms and JavaScript best practices in its code suggestions.

Chapter~\ref{chapter:framework} represents a continuation of our previous work in chapter~\ref{chapter:methodology} by introducing a taxonomy of software abstraction hierarchy to delineate the limitations of \cct{} like Copilot. We also show that Copilot stands at correctness level of our taxonomy.
Finally, we discussed how \cct{} like Copilot can reach the highest level of software abstraction in our taxonomy~(design level). 

The possible applications of large LLMs like Codex are numerous. 
For instance, it might ease users' transition to new codebases, reduce the need for context switching for seasoned programmers, let non-programmers submit specifications, have Codex draught implementations, and support research and education.

GitHub's Copilot and related large language model approaches to code completion are promising steps in \AISE{}.
However, Software systems need more than coding effort. 
These systems require complex design and engineering work to build. 
We showed that while the coding syntax and correctness level of software problems is well on their way to useful support from \cct{} like Copilot, the more abstract concerns, such as code smells, language idioms, and design rules, are far from solvable at present.
% The paper explored potential implications and ways to address these higher abstraction concerns. 
Although far off, we believe \AISE{}, where an AI supports designers and developers in more complex software \emph{development} tasks, is possible.
% API-driven development for recommendations 