\section{\cct{} to design level}
\label{cs2design}
% \subsection{Design Patterns}
% \label{patterns}
Design is typically viewed in the context of software engineering as both a process~\cite{design} that a development team engages in and the specifications~\cite{designdef} that the team produces. 
To satisfy design level, \cct{} should be capable of capturing design and module level concerns. 
These include capturing design patterns~(such as Observer) and architectural tactics~(such as Heartbeat) to improve and personalize suggestions.
The general understanding of a system's design that a software developer has is frequently susceptible to "evaporation," which causes the developers to gradually lose knowledge of the design over time~\cite{martinse} making the process of gathering data to train \cct{} a significant challenge.

Over the natural evolution of a software system, small changes accumulate, which can happen for various reasons, such as refactoring~\cite{fabio}, bug fixes~\cite{cotroneo}, implementation of new features, etc.
These changes can be unique. However, they frequently repeat themselves and follow patterns~\cite{changes}. 
Such patterns can provide a wealth of data for studying the history of modifications and their effects~\cite{martinchanges}, modification histories of fault fixes~\cite{daniel}, or the connections between code change patterns and adaptive maintenance~\cite{ijece}.
However, to use this data, \cct{} should be able to identify these complex patterns existing in public code~(training data). Current \cct{} like Copilot struggled to detect much simpler patterns like Pythonic idioms.

Additionally, current \cct{} like Copilot does not support multi-file input, So it is not possible to evaluate its current performance in design suggestions, as the software development process may include multiple folders with a file structure. 

\cct{} should be able to adapt their suggestions to context-specific issues such as variable naming conventions and formatting. 
This would be challenging as the existing guidelines are not standard in this space and mostly depend on context.

\subsection{Evolution of design over time}
\label{evolution}
Design practices evolve, and \cct{} need to update their suggestions regularly to reflect the design practices changes. 
For example, in JavaScript, callback API was considered the best practice in the past to achieve concurrency, which was replaced by promises. 
While testing, Copilot suggested code specifically mentioned in the JavaScript documentation as a common bad practice and anti-pattern\footref{docs}.

Bad Practices in using promises for asynchronous JavaScript like not returning promises after creation, forgetting to terminate chains without catch statement, which are explained in documentation\footnote{\label{docs}\url{https://developer.mozilla.org/en-US/docs/Web/JavaScript/Guide/Using_promises}} and StackOverflow\footnote{\url{https://stackoverflow.com/questions/30362733/handling-errors-in-promise-all/}} are not known to Copilot and suggested code with those common anti-patterns as they could have occurred more frequently in Copilot training data. 
However, this is beyond the scope of this study and will be part of future work.