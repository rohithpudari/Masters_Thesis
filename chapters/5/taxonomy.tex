\section{Taxonomy}
\label{taxonomy}
Our taxonomy is a software abstraction hierarchy where `basic programming functionality' such as code compilation and syntax checking is at the least abstract level.
Software architecture analysis and design is at the most abstract level.
As we ascend the levels, just as with Koopman's pyramid in Figure \ref{fig:koopman_pyramid}, software challenges rely on successfully solving the challenges shown in Section~\ref{challenges}, and become more difficult to automate (e.g., crafting design rules vs code smells). 

Figure~\ref{fig:taxonomy} shows the taxonomy of autonomy levels for \cct{}.  The more abstract top levels depend on resolution of lower ones. As we move up the hierarchy, we require more human oversight of the AI; as we move down the hierarchy, rules for detecting problems are easier to formulate. Green levels are areas where \AIDE{} works reasonably well, while red levels are challenging for Copilot as shown in Section~\ref{chapter:idioms}.

The challenges further up the hierarchy are nonetheless more important for software quality attributes (QA) \cite{Ernst2017} and for a well-engineered software system.
For example, an automated solution to the top level of the taxonomy would be able to follow heuristics to engineer a well designed software system, one which would be easy to modify and scale to sudden changes in use.

\begin{figure}[hbt!]
    \centering
    \includegraphics[width=\linewidth]{Figures/taxonomy.png}
    \caption{Hierarchy of software abstractions.}
    \label{fig:taxonomy}
\end{figure}
