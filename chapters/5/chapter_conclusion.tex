\section{Conclusion}
\label{conclusion}

% Furthermore, we provided implications for our study. Additionally, we recommend moving beyond token-level suggestions and using reasoning methods to make \cct{} reach the highest level of software abstraction in our taxonomy~(design level).
% Finally, we discussed the threats to the validity of our research and the steps taken to address those threats. We concluded by presenting the future directions of our study. 

Chapter~\ref{chapter:methodology} has shown the current challenges of \cct{} like security issues and license infringements. 
We also showed that \cct{} like Copilot struggle to use Pythonic idioms and JavaScript best practices in its code suggestions.

Chapter~\ref{chapter:framework} represents a continuation of our previous work in chapter~\ref{chapter:methodology} by introducing a taxonomy of software abstraction hierarchy to delineate the limitations of \cct{} like Copilot. We also show that Copilot stands at correctness level of our taxonomy.
Finally, we discussed how \cct{} like Copilot can reach the highest level of software abstraction in our taxonomy~(design level). 

The possible applications of large LLMs like Codex are numerous. 
For instance, it might ease users' transition to new codebases, reduce the need for context switching for seasoned programmers, let non-programmers submit specifications, have Codex draught implementations, and support research and education.

GitHub's Copilot and related large language model approaches to code completion are promising steps in \AISE{}.
However, Software systems need more than coding effort. 
These systems require complex design and engineering work to build. 
We showed that while the coding syntax and correctness level of software problems is well on their way to useful support from \cct{} like Copilot, the more abstract concerns, such as code smells, language idioms, and design rules, are far from solvable at present.
% The paper explored potential implications and ways to address these higher abstraction concerns. 
Although far off, we believe \AISE{}, where an AI supports designers and developers in more complex software \emph{development} tasks, is possible.
% API-driven development for recommendations 