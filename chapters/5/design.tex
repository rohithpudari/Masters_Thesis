\subsection{Design}
\label{design}
Design level is the top level of software abstraction in our taxonomy This level requires the suggested code suggest rational design practices in its code suggestions for a problem and satisfy all the previous levels of abstractions. Design practices are dependent on many factors like requirements, \cct{} should be capable of considering all the relevant factors before suggesting a design practice.

For example, considering the task of performing a sorting operation on a list of numbers. To satisfy this level of abstraction, \cct{} should suggest a syntactically correct list sorting code, using an algorithm which gives the best performance for that particular input scenario like suggesting quick sort algorithm~(avg time complexity = nlogn) instead of bubble sort algorithm~(avg time complexity = n\textsuperscript{2}).

The main goal of this level in the taxonomy is for a \cct{} to be able to pick algorithms and suggest design choices as per given scenario in its suggestions.
The capabilities required by a \cct{} to satisfy this level of abstraction are as follows
\begin{enumerate}
    \item Identify design and module level concerns in code files.
    \item Suggest design patterns and architectural tactics when prompted.
    \item Picking and suggesting the best applicable algorithm for a given scenario.
    \item Satisfy requirements of all previous levels of abstractions.
\end{enumerate}

% \begin{tcolorbox}[title=Code smells level suggestion for sort routine,boxsep=.15mm]
%     %https://tex.stackexchange.com/questions/337909/tcolorbox-tcbline-style
% \textbf{Human Input:}
% \begin{lstlisting}[language={Python}]
% # sorting a list
% arr = [2,4,1,3,7,5]
% \end{lstlisting}
% \tcbline
% \textbf{\cct{} suggestion:}
% \begin{lstlisting}[language={Python}, morekeywords={False, True}]
% swapped = False
% for i in range(len(arr)-1, 0, -1):
% 	for j in range(n):
% 		if arr[j] > arr[j+1]:
% 		    swapped = True
% 			arr[j], arr[j+1] = arr[j+1], arr[j]
% 	if not swapped:
% 	    break
% print(arr)
% \end{lstlisting}
% \end{tcolorbox}