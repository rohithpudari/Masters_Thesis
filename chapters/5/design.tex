\subsection{Design}
\label{design}
The top level in our taxonomy is Design, This level requires the suggested code satisfy all the previous levels of abstractions and suggest appropriate design practices for a given input in its suggestions.

For example, considering the task of sorting operation on a list of numbers, to satisfy this level of abstraction, \cct should suggest a syntactically correct list sorting code, using a algorithm which gives the best performance for that scenario like suggesting quick sort instead of bubble sort if it is a large set of numbers. The main goal of this level in the taxonomy is for a \cct{} to be able to pick the algorithms as per given scenario in its suggestions.

\subsubsection{Design Patterns}
\label{patterns}
To satisfy design level \cct{} must be able to capture design and module level concerns. 
These include recapturing design patterns (such as Observer) and architectural tactics (such as Heartbeat) to improve and personalize suggestions. 
To do this, however, the training data must also use these patterns and best practices.
The vision is for something like \cop{} to be capable of suggesting patterns like Model-View-Controller (MVC) when prompted. 

The capabilities required by a \cct{} to satisfy this level of abstraction are as follows
\begin{enumerate}
    \item Identify design and module level concerns in code files.
    \item Suggest design patterns and architectural tactics when prompted.
    \item Picking and suggesting the best applicable algorithm for a given scenario.
    \item Satisfy requirements of all previous levels of abstractions.
\end{enumerate}