\section{Chapter Summary}
In summary, we start this chapter by showing the methodology used in addressing \textbf{RQ-1.1} (How do \cct{} manage programming idioms?). We first introduced the study setup with the input to Copilot and how it was restricted to deriving the idiomatic way from the input and how the suggestions from Copilot were evaluated. We sampled pythonic idioms from Alexandru et al.~\cite{Alexandru2018} and then compared it against Copilot suggestions. Based on results shown in Table~\ref{tab:all_idioms}, Copilot struggles to suggest the idiomatic ways in its suggestions. 

In this chapter, we showed that Copilot struggles to detect and most common idiomatic ways present in public repositories of GitHub and rank them higher than the non-idiomatic ways. The ideal behavior of \cct{} like Copilot in solving this problem is detecting common patterns present in code and rank them higher as the idiomatic ways for a task.
In the next chapter (chapter~\ref{chapter:smells}), we look into how this ideal behavior can cause problems in the case of code smells, where common bad practices present in public repositories of GitHub can make \cct{} like Copilot introduce bad coding practices in its suggestions.