\section{Chapter Summary}
In summary, we start this chapter by showing the methodology used in addressing textbf{RQ-1} (What are the current boundaries of \cct{}?). 
We first introduced Pythonic idioms and best practices in JavaScript.
We then present our sampling approach for sampling 25 coding scenarios to analyze Copilot code suggestions.
Furthermore, we discussed the input given to Copilot to trigger a code suggestion 
and how the input was restricted to deriving the desired way from the input.
Finally, we described our evaluation approach for Copilot code suggestions.

We sampled 25 Pythonic idioms from Alexandru et al.~\cite{Alexandru2018}, and Farook et al.~\cite{idioms}.
We identified that Copilot did not suggest the idiomatic way as its top suggestion for 23 out of 25 coding scenarios in Python, which addressed \textbf{RQ-1.1} (How do \cct{} manage programming idioms?).
Furthermore, we sampled 25 best practices in JavaScript from the AirBNB JavaScript coding style guide~\cite{airbnb_code}. We identified that Copilot did not suggest the recommended best practice for 22 out of 25 coding scenarios in JavaScript, which addressed \textbf{RQ-1.2} (How do \cct{} manage to manage to suggest non-smelly code?).


% we showed that Copilot struggles to detect and most common idiomatic ways present in public repositories of GitHub and rank them higher than the non-idiomatic ways. The ideal behavior of \cct{} like Copilot in solving this problem is detecting common patterns present in code and rank them higher as the idiomatic ways for a task.
% In the next chapter (chapter~\ref{smells}), we look into how this ideal behavior can cause problems in the case of code smells, where common bad practices present in public repositories of GitHub can make \cct{} like Copilot introduce bad coding practices in its suggestions.

% % \section{Chapter Summary}
% In summary, we start this chapter by showing the methodology used in addressing \textbf{RQ-1.2} (How do \cct{} manage to suggest non-smelly code?). We first introduced the study setup with the input to Copilot and how it was restricted to deriving the best practice from the input and how the suggestions from Copilot were evaluated. We sampled best practices from AirBNB JavaScript coding style guide~\cite{airbnb_code}, and then compared it against Copilot suggestions. Based on results shown in Table~\ref{tab:all_bp}, Copilot struggles to suggest the best practices from widely used coding standards in its suggestions. 

In this chapter, we showed that Copilot struggles to detect and follow coding style guides present in public repositories of GitHub and always suggests code that follows those coding style guides. We also observed that Copilot struggles to detect and most common idiomatic ways present in public repositories of GitHub and rank them higher than the non-idiomatic ways. 
Identifying this delineation could help in urn AI-supported code completion tools such as Copilot into full-fledged AI-supported software engineering tools.
In the next chapter (chapter~\ref{chapter:framework}), we illustrate our taxonomy inspired by autonomous driving levels on the software abstraction hierarchy in \AISE{} and delineate where \cct{} like Copilot currently stands in the taxonomy. 