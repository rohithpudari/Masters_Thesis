\startchapter{Challenges with Copilot}
\label{chapter:methodology}

\section{Introduction}
Useful \cct{} should always suggest recommended coding best practices in its first suggestion. In this chapter, we test if Copilot suggests the optimal solution for a programming task. ``Optimal ways'' here means recommended ways of solving a programming task sampled from popular sources.
We begin by explaining the current challenges with \cct{} like Copilot, showing recent research works on common problems faced with using Copilot and the motivation to find the limitations of current \cct{} like Copilot~(section~\ref{challenges}).

In section~\ref{methodology}, we explain our approach to \textbf{RQ-1} (What are the current boundaries of \cct{}?). 
We describe our sampling approach to collecting Pythonic idioms~(section~\ref{sampling}) and best practices in JavaScript~(section~\ref{smells:sampling}). We then describe the input given to Copilot for triggering the generation of code suggestions~(section~\ref{input}).
Finally, we explain our evaluation method to compare Copilot suggestions to the recommended practices~(section~\ref{evaluation}).

In section~\ref{results}, we present our results on performance of Copilot in suggesting recommended practices for 50 different coding scenarios~(25 Pythonic idioms + 25 code smells), which answers \textbf{RQ-1.1} (How do \cct{} manage programming idioms?), and \textbf{RQ-1.2} (How do \cct{} manage to write non-smelly code?).
We observe that Copilot had the recommended practices in its top 10 suggestions for 18 out of 50 coding scenarios~(36\% of all tests performed).

% In section~\ref{methodology}, we describe our sampling approach to collecting python idioms and the evaluation method used to compare Copilot suggestions to the optimal way listed in python idioms. In section~\ref{secidioms}, we show the results of the study comparing Copilot suggestions and idioms in python programming language. We observe that Copilot performs poorly in suggesting the optimal way in its suggestions. 

% Having identified the performance of current \cct{} like Copilot on detecting and suggesting common patterns like language idioms in chapter~\ref{chapter:idioms}, in this chapter we test if Copilot suggests code that follows code review standards.
% We begin by explaining our approach to \textbf{RQ-1.1} (How do \cct{} manage to write non-smelly code?). To achieve this, we conduct an exploratory study to find if \cct{} tools like Copilot suggest the best practices based on code review standards in their suggestions. 

% In section~\ref{smells:methodology}, we describe our sampling approach to collecting best practices in javascript and the evaluation method used to compare Copilot suggestions to the best practices listed in coding style and code review standards. 
% In section~\ref{bp}, we show the results of the study comparing Copilot suggestions and code review standards in javascript.

\section{Challenges with \cct{}}
\label{challenges}
Code completion tools are very useful, but are often limited to the generation of single elements (e.g. method calls and properties) and the usage of templates. Furthermore, too many recommendations can decrease the usefulness of the tool~\cite{Proksch2015}. 
Some of these basic programming challenges have been already documented and are, we suspect, very much under consideration by the corporate teams behind Copilot and Codex. Since these tools are trained on existing software source code, and training costs are expensive, several classes of errors have been discovered, which follow from the presence of these same errors in public (training) data.

Copilot can make simple coding mistakes, such as not allowing for an empty array in a sort routine\footnote{all examples are documented in our \repl{}.}. Copilot does not understand security vulnerabilities, so it will suggest code that allows for a \textsf{log4shell} vulnerability\footnote{\url{https://www.wiz.io/blog/10-days-later-enterprises-halfway-through-patching-log4shell/}}, or common SQL injection attacks. A recent study by Pearce et al.~\cite{copilot_security} showed that approximately 40\% of the code suggested by Copilot is found to be vulnerable, when tested on 89 different scenarios for Copilot to complete.
Similarly, concerns have been raised about Copilot licence compliance and copyright violation~\cite{code_clone}; with similar input data, Copilot suggests identical code to existing code on GitHub, which may be under copyright. 

% Karampatsis et al.~\cite{github_bugs} showed that for the 1000 most popular open-source Java repositories on GitHub, there is a frequency of one single statement bug per 1600-2500 lines of code and about 33\% of all the bugs match a set of 16 bug templates. This shows that there are bugs which occur repeatedly in the public repositories of GitHub, which can make Copilot biased to suggest bug prone code over bug-free versions.
As it is trained on public data collected in May 2020 from 50 million public repositories on GitHub~\cite{copilot}, any code uploaded after that date is absent in the knowledge base of Copilot. 
% The training data for Copilot was collected in May 2020 from 50 million public repositories on GitHub~\cite{copilot}. 
%It does not have any data uploaded to GitHub after that date and any data from useful sources like documentation and Stack Overflow to improve its suggestions from commonly occurring bugs.
Any software flaws present in large numbers on GitHub will tend to dominate the learning of the model.

But these challenges are not surprising, and have straightforward fixes. These Fixes might include better data engineering and filtering, to remove known problems like a filter that was introduced by GitHub to suppress Copilot suggestions containing code that matches public code on GitHub, although the exact filtering process has not been publicly disclosed.
Similarly, it seems viable to conduct security scans or linting runs prior to making a suggestion, in order to remove obvious problems like SQL injection. 
Clone detection techniques can help find places where code violates copyright. 
Better machine learning approaches, using active learning or fine-tuning, might help learn local lessons~\cite{Menzies2013} for customization in the case of identifier naming or formatting.
In most of these cases, good tools exist already for this. 

Our belief is that while heuristic, such flaws can be fixed and will be fixed in short order. 
What we believe is harder to fix will be problems where straight-forward corrections may not exist and rules for finding problems are harder to specify than those in smell detectors or linters~\cite{Ernst2017} like language idioms and code smells.

\section{Methodology}
\label{methodology}
In this section, We explain the methodology we used to address \textbf{RQ-1} (What are the current boundaries of \cct{}?). We perform our experiments Copilot suggestions on Pythonic Idioms~(section~\ref{idioms}) and code smells in JavaScript~(section~\ref{smells}).

Additionally, we explain how 25 coding scenarios for Pythonic idioms~(section~\ref{smells:sampling}) and code smells in JavaScript~(section~\ref{sampling}) were sampled.
Finally, we discuss how the input is shaped to trigger Copilot to generate code suggestions~(section~\ref{input}) and how Copilot suggestions are evaluated (section~\ref{evaluation}).
The following analysis was carried out using the Copilot extension in Visual Studio Code. We use the most recent stable release of the Copilot extension available at the time of writing~(version number 1.31.6194) in Visual Studio Code.

\subsection{Pythonic Idioms}
\label{python}
A software language is more than just its syntax and semantics; it is also a set of known effective ways to address real-world issues using it. 
To answer \textbf{RQ-1.1} (How do \cct{} manage programming idioms?),
we chose Python, one of the most popular programming languages, because Copilot's base model Codex performs best in Python~\cite{copilot}.  

The definition for the term \emph{Pythonic} in Python found in official Python glossary\footnote{\url{https://docs.python.org/3/glossary.html\#term-pythonic}} as follows:

\begin{quote}
    An idea or piece of code follows the most common idioms of the Python language rather than implementing code using concepts common to other languages. For example, a common idiom in Python is to loop over all elements of an iterable using a for statement. Many other languages do not have this construct, so people unfamiliar with Python sometimes use a numerical counter instead, instead of the cleaner, pythonic method.
\end{quote}

This definition indicates a broad meaning, referring to both concrete code and also \emph{Ideas} in a general sense. Many Python developers argue that coding the \emph{pythonic way} is the most accepted way to code by the Python community~\cite{Alexandru2018}. 
We consider an \emph{idiom} to be any reusable abstraction that makes Python code more readable by shortening or adding syntactic sugar. Idioms can also be more efficient than a basic solution, and some idioms are more readable and efficient.
The pythonicity of a piece of code stipulates how concise, easily readable, and generally good the code is. This concept of pythonicity, as well as the concern about whether code is pythonic or not, is notably prevalent in the Python community.

We sampled idioms from the work of Alexandru et al.~\cite{Alexandru2018}, and Farook et al.~\cite{idioms}, which identified idioms from presentations given by renowned Python developers that frequently mention idioms, e.g., Hettinger~\cite{hettinger} and Jeff Knupp~\cite{knupp} and 
popular Python books, such as ``Pro Python''~\cite{Alchin2010}, ``Fluent Python''~\cite{fluent}, ``Expert Python Programming''~\cite{expert}.


\subsubsection{Sampling Approach}
\label{sampling}
We sampled the top 25 popular pythonic idioms found in open source projects based on the work of Alexandru et al.~\cite{Alexandru2018}, and Farook et al.~\cite{idioms}.
The decision to sample \emph{most popular} pythonic idioms is taken to give the best chance for Copilot to suggest the pythonic way as its top suggestion. As a result, Copilot will have the pythonic way more frequently in its training data and more likely to suggest the pythonic way in its suggestions.
However, Copilot is closed source, and we cannot determine if the frequency of code snippets in training data affects Copilot's suggestions. Research by GitHub shows that Copilot can sometimes recite from its training data in ``generic contexts"\footnote{\url{https://github.blog/2021-06-30-github-copilot-research-recitation/}}, which may lead to potential challenges like license infringements~(shown in section~\ref{challenges}). 
Sampling the most frequently used idioms will also help understand if Copilot can recite idioms present in its training data~(GitHub public repositories), which is a desirable feature for \cct{}.

\subsection{Code Smells}
\label{smells}
A standard style guide is a set of guidelines that explain how code should be written, formatted, and organised. 
Using a style guide ensures that code can be easily shared among developers. As a result, any new developer may immediately become familiar with a specific piece of code and write code that other developers will quickly and easily comprehend.
A good \cct{} tool should only suggest code that is consistent with coding style and passes code reviews by humans. 

To answer \textbf{RQ-1.2} (How do \cct{} manage manage to suggest non-smelly code?), we chose JavaScript, to generalise our experiments with Copilot. 
We relied on the AirBNB javascript coding style guide~\cite{airbnb_code}, a widely used coding style and code review standard introduced in 2012, which is described as ``mostly reasonable approach to JavaScript''~\cite{airbnb_code}.


% \section{Methodology}
% \label{smells:methodology}
% In this Section, we explain the methodology we used to address \textbf{RQ-1.2} (How do \cct{} manage to suggest non-smelly code?), including how the best practices were sampled (section~\ref{smells:sampling}), what was the input for Copilot (section~\ref{smells:input}) and how the Copilot suggestions are evaluated (section~\ref{smells:evaluation}). All of the following analysis was carried out using Copilot extension in visual studio code. We use the most recent stable release of Copilot extension (version number 1.30.6165) in visual studio code.

\subsubsection{Sampling Approach}
\label{smells:sampling}
The AirBNB JavaScript coding style guide~\cite{airbnb_code} contains a comprehensive list of best practices covering nearly every aspect of javascript coding like objects, arrays, modules, and iterators. However, it also includes project specific styling guidelines like naming conventions, commas, comments.
Since, we are testing Copilot for widely accepted best practices and not project specific styling in JavaScript. 
We sampled 25 best practices from the AirBNB JavaScript coding style guide~\cite{airbnb_code}, 
which were closer to design level rather than code level. For example, selecting logging practices as a sample coding standard rather than trailing comma use in javascript as a coding standard. 
This sampling approach is done to ensure Copilot is not tested against personalised styling guidelines of one specific project or a company. In contrast, our goal for Copilot here is to be tested against practices that actually bring performance or efficiency to the code base.

% \subsection{Study Setup}

% \subsection{Input to Copilot}
% \label{smells:input}
% The input to Copilot consisted of the best practice title as the first comment to provide context, and the input was restricted to being able to derive the best practice from the input. This is done to ensure Copilot is making the decision to suggest the good/bad way in its suggestions. This input style also mimics a novice user, who is unaware of the best practices in coding style guides and useful \cct{} should drive the novice user to use best practices.

\section{Input to Copilot}
\label{input}
The input to Copilot consisted of the idiom title in the first line as a comment to provide context, and the input was restricted to being able to derive the ideal way~(idiomatic way) from the input. This is done to ensure Copilot is making the decision to suggest the good/bad way in its suggestions and not being restricted by the input to suggest a certain way. 

This input style also mimics a novice user, who is unaware of the idioms and useful \cct{} should drive the novice user to use idiomatic ways to perform a task in their codebases.
\subsection{Evaluation of Copilot suggestions}
\label{evaluation}
We compare Copilot code suggestions against Pythonic idioms and best practices retrieved from our sources~(Alexandru et al.~\cite{Alexandru2018} and Farook et al.~\cite{idioms} for Pythonic idioms and AirBNB javascript coding style guide~\cite{airbnb_code} for JavaScript code smells). If Copilot manages to match the Pythonic idiom or the best practice as its first suggestion, we considered Copilot to suggest the desired approach and passed the coding scenario. 
In contrast, if Copilot did not have Pythonic idiom or the best practice in any of its all 10 code suggestions currently viewable using Copilot extension in Visual Studio Code, we considered Copilot did not suggest the desired approach and failed the coding scenario.

We assume that \cct{} like Copilot are productivity tools, and the user should be saving time as opposed to writing the optimal way without using \cct{}, scrolling through all the suggestions to deduce the idiomatic approach or the best practice that follows the coding style guide defeats this purpose. 
For this reason, We restricted ourselves to the first suggestion of Copilot to be considered in determining the Pass/Fail status of the coding scenario. However, we note if the best practice appeared in any of its ten suggestions.
\section{Results}
\label{chapter:results}

% \section{Introduction}
% Having identified the performance of current \cct{} like Copilot on detecting and suggesting common patterns like language idioms in chapter~\ref{chapter:idioms}, in this chapter we test if Copilot suggests code that follows code review standards.
% We begin by explaining our approach to \textbf{RQ-1.1} (How do \cct{} manage to write non-smelly code?). To achieve this, we conduct an exploratory study to find if \cct{} tools like Copilot suggest the best practices based on code review standards in their suggestions. 

In section~\ref{smells:methodology}, we describe our sampling approach to collecting best practices in javascript and the evaluation method used to compare Copilot suggestions to the best practices listed in coding style and code review standards. 
In section~\ref{bp}, we show the results of the study comparing Copilot suggestions and code review standards in javascript.

\subsection{Pythonic Idioms}
\label{idioms}
Using the sampling approach described in section~\ref{sampling}, we sampled the 25 most popular Python idioms from the work of Alexandru et al.~\cite{Alexandru2018}, and Farook et al.~\cite{idioms}. 
We then compared Copilot suggestions when prompted with an input~(shown in section~\ref{input}) to trigger a code suggestion from Copilot and present our results using the evaluation approach~(shown in section~\ref{evaluation}).

Copilot suggested the idiomatic approach as the first suggestion in 2 of the 25 idioms we tested, i.e., 2 out of 25 instances, Copilot had the recommended idiomatic approach as its top suggestion. 
However, 8 out of those remaining 23 Idioms had the idiomatic way in Copilot's top 10 suggestions. Copilot did not have the idiomatic way in its top 10 suggestions for 15 idioms out of 25 we tested.

The results show that Copilot did not suggest the optimal way as its first suggestion in the majority~(92\%) of the idioms we tested. This indicates that current \cct{} like Copilot cannot suggest the idiomatic way even though they are the top most frequently used Python idioms in public repositories on GitHub~\cite{Alexandru2018, idioms}. 

Copilot being closed source, we cannot investigate the potential reasons behind this behavior. However, one plausible explanation for this behavior is that idiomatic ways may not be as frequent as non-idiomatic ways in Copilot's training data of public repositories on GitHub, making the non-idiomatic way rank higher than the idiomatic way.

\begin{figure}[hbt!]
    \centering
    \includegraphics[width=\linewidth]{Figures/idioms_1.png}
    \caption{List Comprehension Idiom and Copilot Suggestion}
    \label{fig:idioms_1}
\end{figure}

Figure~\ref{fig:idioms_1} shows the example of list comprehension Pythonic idiom, showing user input (i.e., human input), the top suggestion by Copilot, and the idiomatic way from Alexandru et al.~\cite{Alexandru2018}.

\cct{} like Copilot should learn to detect idiomatic ways in public repositories and rank them higher than the most frequently used way in public repositories so that the first suggestion would be the idiomatic way rather than the non-idiomatic way, which is the desired behavior for \cct{} like Copilot. For the scope of this thesis, we leave resolving this problem as future work.

% \begin{tcolorbox}[title=List Comprehension,boxsep=.25mm]
%     %https://tex.stackexchange.com/questions/337909/tcolorbox-tcbline-style
% \textbf{Human Input:}
% \begin{lstlisting}[language={Python}]
% #list comprehension
% result_list = 
% \end{lstlisting}
% \tcbline
% \textbf{Copilot Suggestion:}
% \begin{lstlisting}[language=Python,escapechar=\%]
% % \noindent\textcolor{gray}{result\_list  =} % []
% for i in range(1,11):
%     result_list.append(i)
% \end{lstlisting}
% \tcbline
% \textbf{Idiomatic way\footnote{source \cite{Alexandru2018}}:}
% \begin{lstlisting}[language=Python]
% result_list = [el for el in range(11)]
% \end{lstlisting}
% \end{tcolorbox}

%%%%%% TODO: remember to update the screenshot if the source citation is different from the citation in the text %%%%%%
Table~\ref{tab:all_idioms} shows the list of all the 25 idioms we tested and the ranking of the idiomatic way in Copilot suggestions (if it exists).
All the Idioms shown in Table~\ref{tab:all_idioms} can be found in the \repl{} including the code used as input (i.e., human input), the top suggestion by Copilot, and the idiomatic way suggested in Alexandru et al.~\cite{Alexandru2018}, and Farook et al.~\cite{idioms}. 

\renewcommand{\arraystretch}{1.45}
\begin{table}[hbt!]

    \begin{tabular}{|c|c|c|}
        \hline
        \centering
         \textbf{S No.} & \textbf{Idiom Title} & \textbf{Copilot Suggestion Matched?}  \\
         & & (out of 10 suggestions) \\
         \hline
         1 & List comprehension & No \\
         \hline
         2 & Dictionary comprehension & No \\
         \hline
         3 & Mapping & 9\textsuperscript{th} \\
         \hline
         4 & Filter &  7\textsuperscript{th} \\
         \hline
         5 & Reduce & 9\textsuperscript{th} \\
         \hline
         6 & List enumeration & No \\
         \hline
         7 & Set comprehension & 1\textsuperscript{th} \\
         \hline
         8 & Read and print from a file & 5\textsuperscript{th} \\
         \hline
         9 & Add int to all list numbers & No \\
         \hline
         10 & If condition check value & 1\textsuperscript{th} \\
         \hline
         11 & Unpacking operators & No \\
         \hline
         12 & Open and write to a file & 6\textsuperscript{th} \\
         \hline
         13 & Access key in dictionary & No \\
         \hline
         14 & Print variables in strings & No \\
         \hline
         15 & Index of every word in input string & No \\
         \hline
         16 & Boolean comparision & 2\textsuperscript{nd} \\
         \hline
         17 & Check for null string & 5\textsuperscript{th} \\
         \hline
         18 & Check for empty list & 4\textsuperscript{th} \\
         \hline
         19 & Multiple conditions in if statement & No \\
         \hline
         20 & Print everything in list & No \\
         \hline
         21 & Zip two lists together & No \\
         \hline
         22 & Combine iterable separated by string & No \\
         \hline
         23 & Sum of list elements & No \\
         \hline
         24 & List iterable creation & No \\
         \hline
         25 & Function to manage file & No \\
         \hline
    \end{tabular}
    \caption{List of all Pythonic idioms tested on Copilot.}
    \label{tab:all_idioms}
\end{table}
\subsection{Code Smells}
\label{smells:results}
Using the sampling approach described in section~\ref{sampling}, we sampled 25 best practices in JavaScript from the AirBNB JavaScript coding style guide~\cite{airbnb_code}. 
We sampled best practices that are closer to the design level rather than the code level. For example, selecting logging practices as a sample coding standard rather than trailing comma use in JavaScript as a coding standard.

We then compared Copilot suggestions when prompted with an input(shown in section~\ref{input}), which includes the title of the coding scenario being tested as the first line to provide context for Copilot and help trigger a code suggestion. 
We then present our results using the evaluation approach~(shown in section~\ref{evaluation}).

Copilot suggested the best practice from the AirBNB JavaScript coding style guide~\cite{airbnb_code} for 3 out of the 25 coding standards we tested, i.e., 3 out of 25 instances Copilot had the recommended best practice as its top suggestion.
Moreover, only 5 of the remaining 22 coding scenarios had the best practice in Copilot's top 10 suggestions currently viewable. 
Copilot did not have the best practice in its top 10 suggestions for 17 scenarios out of 25 coding scenarios we tested.

The results show that Copilot did not suggest the recommended best practice as its first suggestion in the majority (88\%) of the best practices we tested.
As Copilot is closed source, we cannot find the reason behind this, but one could argue that lack of data for JavaScript compared to Python could be a reason for this behavior. 

% The results show that Copilot performed worse than the language idioms. This indicates that current AI-supported
% code completion tools like Copilot are not yet capable of suggesting the best practices in their suggestions, even though the best practices are sampled from a widely accepted coding style guide.

% There could be many reasons for this performance, like the public repositories do not always follow coding standards, and Copilot cannot detect coding styles from repositories with contribution guides, including the coding standards followed in the project. 
% Copilot being closed source, we cannot investigate the potential reasons behind this behavior and recommend ways to fix this issue, improving the performance of Copilot.

We did not test Copilot for suggesting project-specific coding styles because Copilot does not have the feature to customize its suggestions based on preferences~\cite{Copilot-web}.
However, \cct{} like Copilot should follow coding style guides and adapt their code suggestions to match the coding style used in the project. 
For example, if a user is working on a project where one of the coding style guidelines says to leave a blank line after blocks and before the next statement~\cite{airbnb_code}. As a productivity tool, the ideal behavior for \cct{} like Copilot is to detect the coding style guideline from existing code or a coding style document in the project and always suggest code that follows the guidelines.

Figure~\ref{fig:bp_1} shows the Best Practice for Copying Array Contents, showing user input (i.e., Human Input), the top suggestion by Copilot, and the recommended way suggested by AirBNB JavaScript coding style guide~\cite{airbnb_code}. 
The code suggested by Copilot uses a for loop to iterate through all the contents of the array and copy them, which is a lot slower than the spread approach~([..items]) shown as the best practice to copy array in the AirBNB JavaScript coding style guide~\cite{airbnb_code}. This eliminates the need to iterate through all the elements of the array to copy each one. 

\begin{figure}[hbt!]
    \centering
\begin{tcolorbox}[title=Copying Array Contents,boxsep=.15mm]
    %https://tex.stackexchange.com/questions/337909/tcolorbox-tcbline-style
\textbf{Human Input:}
\begin{lstlisting}[language=JavaScript]
// copying array contents
\end{lstlisting}
\tcbline
\textbf{Copilot Suggestion:}
\begin{lstlisting}[language=JavaScript]
const len = items.length;
const copy = new Array(len);
for (let i = 0; i < len; i++) {
	  copy[i] = items[i];
}
\end{lstlisting}
\tcbline
\textbf{Ideal way\footnote{source \cite{airbnb_code}}:}
\begin{lstlisting}[language=JavaScript]
const itemsCopy = [...items];
\end{lstlisting}
\end{tcolorbox}
    \caption{Best practice for copying array contents and Copilot top suggestion.}
    \label{fig:bp_1}
\end{figure}

Another example for best practices being better than Copilot code suggestions is shown in figure~\ref{fig:bp_2}. 
The AirBNB JavaScript coding style guide~\cite{airbnb_code} recommends to always use `const' for all the references and avoid using `var' because using `const' ensures that you cannot reassign your references, which in turn can lead to bugs and difficulty in code comprehension.
Copilot suggested to use `var' as its first suggestion.
This shows that code suggested by Copilot has flaws and does not follow the recommended best practices.

\begin{figure}[hbt!]
    \centering
\begin{tcolorbox}[title=Create two references,boxsep=.15mm]
    %https://tex.stackexchange.com/questions/337909/tcolorbox-tcbline-style
\textbf{Human Input:}
\begin{lstlisting}[language=JavaScript]
// create two references
\end{lstlisting}
\tcbline
\textbf{Copilot Suggestion:}
\begin{lstlisting}[language=JavaScript]
var a = 1;
var b = 2;
\end{lstlisting}
\tcbline
\textbf{Ideal way\footnote{source \cite{airbnb_code}}:}
\begin{lstlisting}[language=JavaScript]
const a = 1;
const b = 2;
\end{lstlisting}
\end{tcolorbox}
    \caption{Best practice for creating two references and Copilot top suggestion.}
    \label{fig:bp_2}
\end{figure}

Figure~\ref{fig:bp_3} shows the best practice to return a string and a variable name from a function. 
The AirBNB JavaScript coding style guide~\cite{airbnb_code} suggests to use template strings instead of concatenation, for programmatically building up strings because template strings give a readable, concise syntax with proper newlines and string interpolation features.
Copilot suggested concatenation as its first suggestion~(shown in figure~\ref{fig:bp_3}).

\begin{figure}[hbt!]
    \centering
\begin{tcolorbox}[title=Return string and variable name,boxsep=.15mm]
    %https://tex.stackexchange.com/questions/337909/tcolorbox-tcbline-style
\textbf{Human Input:}
\begin{lstlisting}[language=JavaScript]
// return string and variable name
function sayhi(name){
\end{lstlisting}
\tcbline
\textbf{Copilot Suggestion:}
\begin{lstlisting}[language=JavaScript,escapechar=\%]
% \noindent\textcolor{gray}{function sayhi(name)\{ } %
    return "Hello " + name;
}
\end{lstlisting}
\tcbline
\textbf{Ideal way\footnote{source \cite{airbnb_code}}:}
\begin{lstlisting}[language=JavaScript]
function sayHi(name) {
    return "Hello, ${name}";
}
\end{lstlisting}
\end{tcolorbox}
    \caption{Best practice for returning string and variable name and Copilot top suggestion.}
    \label{fig:bp_3}
\end{figure}

Table~\ref{tab:all_bp} shows the complete list of all the best practices we tested on Copilot sampled from the AirBNB Coding Style guide~\cite{airbnb_code} and the ranking of the best practice in Copilot suggestions (if it exists).

All the best practices shown in Table~\ref{tab:all_bp} can be found in the \repl{} including the code used as input (i.e., human input), the top suggestion by Copilot, and the best practice from AirBNB JavaScript coding style guide~\cite{airbnb_code}.

The results show that Copilot performed worse than the language idioms. This indicates that current AI-supported
code completion tools like Copilot are not yet capable of suggesting the best practices in their suggestions, even though the best practices are sampled from a widely accepted coding style guide.

There could be many reasons for this performance, like the public repositories do not always follow coding standards, and Copilot cannot detect coding styles from repositories with contribution guides, including the coding standards followed in the project. 
Copilot being closed source, we cannot investigate the potential reasons behind this behavior and recommend ways to fix this issue, improving the performance of Copilot. However, improving the frequency of best practice usage in training data and including metrics such as repository popularity in ranking of code suggestions could be some potential areas to explore for improving performance of Copilot.

Copilot had the recommended best practice in its top 10 suggestions for 5 coding scenarios, where Copilot did not rank the recommended best practice as the top suggestion. 
The ranking methodology of Copilot is not disclosed. However, the results suggest that it is heavily influenced by the frequency of the approach in the training data. 
Copilot successfully suggested the recommended best practice as its top suggestion in `accessing properties,' `converting an array-like object to an array, and `Check boolean value'~(best practice 7, 15 \& 23 in table~\ref{tab:all_bp}), which are one of the most common practices used by beginners to perform the task~\cite{airbnb_code}.

Based on the results shown in table~\ref{tab:all_bp}, Copilot is more likely to have the recommended best practice in its top 10 suggestions when it is a common beginner programming task like finding `sum of numbers' or `importing a module from a file.' 
We also observed that Copilot did not always generate all 10 suggestions like in the case of Pythonic idioms, and it struggled to come up with 10 suggestions to solve a programming task.
This shows that Copilot does not have enough training data compared to Python to create more relevant suggestions, which may include the recommended best practices in JavaScript.

The ideal behavior for \cct{} like Copilot is to suggest best practices extracted from public code repositories~(training data) to avoid code smells. 
Additionally, \cct{} like Copilot should detect the project's coding style and adapt its code suggestions to be helpful for a user as a productivity tool. 
For the scope of this thesis, we leave resolving this problem as future work.

\begin{table}[hbt!]
        \centering
    \begin{tabular}{|c|c|c|}
        \hline

        \textbf{S No.} & \textbf{Best Practice  Title} & \textbf{Copilot Suggestion Matched?} \\
         & & (out of 10 suggestions) \\
         \hline
         1 & Usage of Object method shorthand & No \\
         \hline
         2 & Array Creating Constructor & 6\textsuperscript{th} \\
         \hline
         3 & Copying Array Contents  & No \\
         \hline
         4 & Logging a Function &  No \\
         \hline
         5 & Exporting a Function & No \\
         \hline
         6 & Sum of Numbers & 9\textsuperscript{th} \\
         \hline
         \textbf{7} & \textbf{Accessing Properties} & \textbf{1\textsuperscript{st}} \\
         \hline
         8 & Switch case usage & No \\
         \hline
         9 & Return value after condition & No \\
         \hline
         10 & Converting Array-like objects  & No \\
         \hline
         11 & Create two references & 5\textsuperscript{th} \\
         \hline
         12 & Create and reassign reference & No \\
         \hline
         13 & Shallow-copy objects  & No \\
         \hline
         14 & Convert iterable object to an array & No \\
         \hline
         \textbf{15} & \textbf{Converting array like object to array} & \textbf{1\textsuperscript{st}} \\
         \hline
          16 & Multiple return values in a function & No \\
          \hline
          17 & Return string and variable name & No \\
          \hline
          18 & Initialize object property & No \\
          \hline
          19 & Initialize array callback & No \\
          \hline
          20 & Import module from file & 6\textsuperscript{th} \\
          \hline
          21 & Exponential value of a number & No \\
          \hline
          22 & Increment a number & 2\textsuperscript{nd} \\
          \hline
          \textbf{23} & \textbf{Check boolean value} & \textbf{1\textsuperscript{st}} \\
          \hline
          24 & Type casting constant to a string & No \\
          \hline
          25 & Get and set functions in a class & No \\
          \hline
    \end{tabular}
    \caption{List of all JavaScript best practices tested on Copilot.}
    \label{tab:all_bp}
\end{table}
\subsection{Summary of findings}
In an attempt to find the boundaries of \cct{} like Copilot, we analyzed Copilot code suggestions for Pythonic idioms and JavaScript best practices. 
We identified that Copilot did not suggest the idiomatic way as its top suggestion for 23 out of 25 coding scenarios in Python. 
Furthermore, we identified that Copilot did not suggest the recommended best practice for 22 out of 25 coding scenarios in JavaScript.

Although Copilot is very good at solving well-specified programming contest style problems~\cite{empirical_eval}, our experiments show that it does not do well in following idioms and recommending best practices in its code suggestions.
Additionally, \cct{} like Copilot being a productivity tool, should be able to suggest idiomatic approaches and recommended best practices in its code suggestions to be helpful for the user.
Studies like ours might help use this delineation to understand what might help turn \cct{} such as Copilot into full-fledged \AISE{} tools.
\section{Chapter Summary}
In summary, we start this chapter by showing the methodology used in addressing textbf{RQ-1} (What are the current boundaries of \cct{}?). 
We first introduced Pythonic idioms and best practices in JavaScript.
We then present our sampling approach for sampling 25 coding scenarios to analyze Copilot code suggestions.
Furthermore, we discussed the input given to Copilot to trigger a code suggestion 
and how the input was restricted to deriving the desired way from the input.
Finally, we described our evaluation approach for Copilot code suggestions.

We sampled 25 Pythonic idioms from Alexandru et al.~\cite{Alexandru2018}, and Farook et al.~\cite{idioms}.
We identified that Copilot did not suggest the idiomatic way as its top suggestion for 23 out of 25 coding scenarios in Python, which addressed \textbf{RQ-1.1} (How do \cct{} manage programming idioms?).
Furthermore, we sampled 25 best practices in JavaScript from the AirBNB JavaScript coding style guide~\cite{airbnb_code}. We identified that Copilot did not suggest the recommended best practice for 22 out of 25 coding scenarios in JavaScript, which addressed \textbf{RQ-1.2} (How do \cct{} manage to manage to suggest non-smelly code?).


% we showed that Copilot struggles to detect and most common idiomatic ways present in public repositories of GitHub and rank them higher than the non-idiomatic ways. The ideal behavior of \cct{} like Copilot in solving this problem is detecting common patterns present in code and rank them higher as the idiomatic ways for a task.
% In the next chapter (chapter~\ref{smells}), we look into how this ideal behavior can cause problems in the case of code smells, where common bad practices present in public repositories of GitHub can make \cct{} like Copilot introduce bad coding practices in its suggestions.

% % \section{Chapter Summary}
% In summary, we start this chapter by showing the methodology used in addressing \textbf{RQ-1.2} (How do \cct{} manage to suggest non-smelly code?). We first introduced the study setup with the input to Copilot and how it was restricted to deriving the best practice from the input and how the suggestions from Copilot were evaluated. We sampled best practices from AirBNB JavaScript coding style guide~\cite{airbnb_code}, and then compared it against Copilot suggestions. Based on results shown in Table~\ref{tab:all_bp}, Copilot struggles to suggest the best practices from widely used coding standards in its suggestions. 

In this chapter, we showed that Copilot struggles to detect and follow coding style guides present in public repositories of GitHub and always suggests code that follows those coding style guides. We also observed that Copilot struggles to detect and most common idiomatic ways present in public repositories of GitHub and rank them higher than the non-idiomatic ways. 
Identifying this delineation could help in urn AI-supported code completion tools such as Copilot into full-fledged AI-supported software engineering tools.

In the next chapter (chapter~\ref{chapter:framework}), we illustrate our taxonomy inspired by autonomous driving levels on the software abstraction hierarchy in \AISE{} and use the results shown in this chapter to delineate where \cct{} like Copilot currently stands in the taxonomy. 