\subsection{Common Errors}
\label{common}

Github Copilot\cite{copilot} training data consists of code written by people from various levels of experience from a student who is just staring to learn how to code to an expert in the field. A recent study \cite{github_bugs} on 1000 most popular open-source java repositories on GitHub, showed a frequency of one single statement bug per 1600-2500 lines of code and about 33\% of all the bugs match a set of 16 bug templates. This shows that there are bugs which occur repeatedly in the public repositories of GitHub and can make copilot biased to suggest bug prone code over bug-free version.

We ran an experiment with copilot to create a popular array sorting algorithm called bubble sort. Using the code suggested by copilot, it missed edge cases like empty arrays.

\begin{tcolorbox}[title=BubbleSort algorithm,boxsep=.5mm]
    %https://tex.stackexchange.com/questions/337909/tcolorbox-tcbline-style
\textbf{Human Input:}
\begin{lstlisting}[language={JavaScript}]
//bubblesort.js
function bubbleSort(array) {
\end{lstlisting}
\tcbline
\textbf{Copilot Suggestion:}
\begin{lstlisting}[language=JavaScript,escapechar=\%]
	var swapped;
  do {
    swapped = false;
    for (var i = 0; i < array.length - 1; i++) {
      if (array[i] > array[i + 1]) {
	var temp = array[i];
	array[i] = array[i + 1];
	array[i + 1] = temp;
	swapped = true;
      }
    }
  } while (swapped);
  return array;
}

function swap(array, i, j) {
  var temp = array[i];
  array[i] = array[j];
  array[j] = temp;
}
\end{lstlisting}
% \tcbline
% \textbf{Ideal way\footnote{source \cite{Alexandru2018}}:}
% \begin{lstlisting}[language=Python]
% result_list = [el for el in range(11)]
% \end{lstlisting}
\end{tcolorbox}