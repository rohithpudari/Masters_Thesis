\subsection{Input to Copilot}
\label{input}
The input for Copilot to trigger code suggestions consists of two parts.
First, the title of the coding scenario is tested as the first line as a comment to provide context for Copilot to trigger relevant suggestions while stating the motive of the code scenario. 
Second, minimal input of code is required to trigger the code suggestion. Moreover, the code input was restricted to being able to derive the best practice from the information. 
This is to ensure Copilot is deciding to suggest the good/bad way in its suggestions and not being restricted by the input to suggest a certain way. 

Copilot does not have the functionality to override or update the input, and it will only suggest code that matches the input. So, it is important to restrict the input to accurately test Copilot without limiting its possibility of creating different coding scenarios, which may include the best practice we desired. 
For example, figure~\ref{fig:idioms_1} shows a example of list comprehension idiom where human input is restricted to just declaring a variable ``result\_list''. If the input included initializing the variable with some integers or an empty list, 
then Copilot is forced to use a for loop in the next line to perform list comprehension eliminating the possibility of suggesting the idiomatic approach. 
Although, it is a desirable feature for \cct{} to override or update the input, Copilot does not support it yet. So, we restrict the input to being able to derive the best practice from the information. 

This input style also mimics a novice user, who is unaware of the idioms or best practices. 
Useful \cct{} like Copilot should drive the novice user to use best practices to perform a task in their codebases and improve the quality of their code.