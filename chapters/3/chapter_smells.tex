\startchapter{Code Smells}
\label{chapter:smells}

\section{Introduction}
The current approaches focus on programming-in-the-small \cite{DeRemer1976} i.e, on individual lines of code. 
Code language models have focused (effectively!) on source code as natural language~\cite{natural}.
This models the software development task as predicting the next token or series of tokens.
As Copilot's webpage says, the aim for Copilot is to produce ``safe and effective code [with] suggestions for whole lines or entire functions'' as ``your AI pair programmer''~\cite{Copilot-web}. 

Since these tools are trained on existing software source code, and training costs are expensive, several classes of errors have been discovered, which follow from the presence of these same errors in public (training) data. In this chapter, we begin by explaining our approach to \textbf{RQ-1.1} (How do \cct{} manage to write non-smelly code?). we categorise the different kinds of errors that have been discovered so far, showing an example for each category. 
We introduce the methodology used to analyse \cop{} code suggestions. (Section~\ref{method}) for the three categories under study: Licence Infringements (Section~\ref{licence}), Potential Security issues (Section~\ref{security}) and Common errors (Section~\ref{common}).

We also discuss on how sensitive \cop{} is to the user input and show potential directions to resolve these issues with recommendations based on the analysing \cop{} code suggestions (Section~\ref{smells:dis}).

\section{Methodology}
\label{method}
In this Section, we explain the methodology we used to address \textbf{RQ-1.1} (How do \cct{} manage to write non-smelly code?), including what was the input for copilot and how is the suggestion evaluated. All of the following analysis was carried out using \cop{} extension in Visual Studio Code, which makes \cop{} model versioning and clarity challenging. We use the most recent stable release of Copilot extension (version number todo) in visual studio code. 

\subsection{Licence Infringements}
We considered input of 10 lines from a random file on github with at least 100 lines of code. If \cop{} suggests next 20 lines exactly like the source file on github, we considered it as a copyright violation. 

\subsection{Security Issues and Common Errors}
The input to \cop{} consisted of the task title as the first comment to provide context, and the input was restricted to being able to derive the ideal way from the input. This is done to ensure Github Copilot is making the decision to suggest the non-ideal/ideal way in its suggestions. This input style also mimics a novice user, who is unaware of the ideal practices and useful AI-supported code completion tools
should drive the novice user to use ideal practices.  

\section{Licence Infringements}
\label{licence}
\subsection{Security Issues}
\label{security}


\subsection{Common Errors}
\label{common}

Github Copilot\cite{copilot} training data consists of code written by people from various levels of experience from a student who is just staring to learn how to code to an expert in the field. A recent study \cite{github_bugs} on 1000 most popular open-source java repositories on GitHub, showed a frequency of one single statement bug per 1600-2500 lines of code and about 33\% of all the bugs match a set of 16 bug templates. This shows that there are bugs which occur repeatedly in the public repositories of GitHub and can make copilot biased to suggest bug prone code over bug-free version.

We ran an experiment with copilot to create a popular array sorting algorithm called bubble sort. Using the code suggested by copilot, it missed edge cases like empty arrays.

\begin{tcolorbox}[title=BubbleSort algorithm,boxsep=.5mm]
    %https://tex.stackexchange.com/questions/337909/tcolorbox-tcbline-style
\textbf{Human Input:}
\begin{lstlisting}[language={JavaScript}]
//bubblesort.js
function bubbleSort(array) {
\end{lstlisting}
\tcbline
\textbf{Copilot Suggestion:}
\begin{lstlisting}[language=JavaScript,escapechar=\%]
	var swapped;
  do {
    swapped = false;
    for (var i = 0; i < array.length - 1; i++) {
      if (array[i] > array[i + 1]) {
	var temp = array[i];
	array[i] = array[i + 1];
	array[i + 1] = temp;
	swapped = true;
      }
    }
  } while (swapped);
  return array;
}

function swap(array, i, j) {
  var temp = array[i];
  array[i] = array[j];
  array[j] = temp;
}
\end{lstlisting}
% \tcbline
% \textbf{Ideal way\footnote{source \cite{Alexandru2018}}:}
% \begin{lstlisting}[language=Python]
% result_list = [el for el in range(11)]
% \end{lstlisting}
\end{tcolorbox}
\section{Discussion}
\section{Chapter Summary}
In summary, we start this chapter by showing the methodology used in addressing textbf{RQ-1} (What are the current boundaries of \cct{}?). 
We first introduced Pythonic idioms and best practices in JavaScript.
We then present our sampling approach for sampling 25 coding scenarios to analyze Copilot code suggestions.
Furthermore, we discussed the input given to Copilot to trigger a code suggestion 
and how the input was restricted to deriving the desired way from the input.
Finally, we described our evaluation approach for Copilot code suggestions.

We sampled 25 Pythonic idioms from Alexandru et al.~\cite{Alexandru2018}, and Farook et al.~\cite{idioms}.
We identified that Copilot did not suggest the idiomatic way as its top suggestion for 23 out of 25 coding scenarios in Python, which addressed \textbf{RQ-1.1} (How do \cct{} manage programming idioms?).
Furthermore, we sampled 25 best practices in JavaScript from the AirBNB JavaScript coding style guide~\cite{airbnb_code}. We identified that Copilot did not suggest the recommended best practice for 22 out of 25 coding scenarios in JavaScript, which addressed \textbf{RQ-1.2} (How do \cct{} manage to manage to suggest non-smelly code?).


% we showed that Copilot struggles to detect and most common idiomatic ways present in public repositories of GitHub and rank them higher than the non-idiomatic ways. The ideal behavior of \cct{} like Copilot in solving this problem is detecting common patterns present in code and rank them higher as the idiomatic ways for a task.
% In the next chapter (chapter~\ref{smells}), we look into how this ideal behavior can cause problems in the case of code smells, where common bad practices present in public repositories of GitHub can make \cct{} like Copilot introduce bad coding practices in its suggestions.

% % \section{Chapter Summary}
% In summary, we start this chapter by showing the methodology used in addressing \textbf{RQ-1.2} (How do \cct{} manage to suggest non-smelly code?). We first introduced the study setup with the input to Copilot and how it was restricted to deriving the best practice from the input and how the suggestions from Copilot were evaluated. We sampled best practices from AirBNB JavaScript coding style guide~\cite{airbnb_code}, and then compared it against Copilot suggestions. Based on results shown in Table~\ref{tab:all_bp}, Copilot struggles to suggest the best practices from widely used coding standards in its suggestions. 

In this chapter, we showed that Copilot struggles to detect and follow coding style guides present in public repositories of GitHub and always suggests code that follows those coding style guides. We also observed that Copilot struggles to detect and most common idiomatic ways present in public repositories of GitHub and rank them higher than the non-idiomatic ways. 
Identifying this delineation could help in urn AI-supported code completion tools such as Copilot into full-fledged AI-supported software engineering tools.
In the next chapter (chapter~\ref{chapter:framework}), we illustrate our taxonomy inspired by autonomous driving levels on the software abstraction hierarchy in \AISE{} and delineate where \cct{} like Copilot currently stands in the taxonomy. 