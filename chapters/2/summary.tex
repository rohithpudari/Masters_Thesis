\section{Chapter Summary}
In this chapter, we first provided some background on different language models used for \cct{} and their limitations. We reviewed early developments in \cct{} with statistical language models like N-grams. Followed by a discussion on N-gram based \cct{} and the limitations of statistical language models, resulting in using neural language models for \cct{}.

We established the importance of transformers for \cct{} which is a crucial component of OpenAI's Codex Model~\cite{copilot}. To further explore the role of transformer architecture in \cct{}, we reviewed studies showing context-sensitive contextualized word representations were presented by LMs such as BERT~\cite{bert}.

We then discussed GitHub Copilot, the \cct{} we would use as the basis for our study in this thesis. Additionally, we reviewed the key functionalities of Copilot.
Furthermore, we discussed some related works on Copilot about its usage~\cite{Vaithilingam2022} and its effectiveness in solving programming contest-style problems~\cite{empirical_eval}. We concluded by introducing some of the other \cct{} that provide similar functionality to Copilot.

In the following chapters, we discuss the problems with using \cct{} like Copilot, which are harder to fix, and straightforward corrections that may not exist, like language idioms and code smells. 
We try to address \textbf{RQ-1} (What are the current boundaries of code completion tools) using the methodology and present our results~(Chapter~\ref{chapter:methodology}).
We then introduce a taxonomy of software abstraction hierarchy to help with finding the current boundaries of \cct{} like Copilot (Chapter~\ref{chapter:framework}). 

We address \textbf{RQ-2} (Given the current boundary, how far is it from suggesting design decisions?) with a discussion of the complex nature of design decisions and the challenges with trying to use \cct{} to make design decisions.
Finally, we discuss some of the practical implications and limitations of our findings and also provide some future directions to help further research in \cct{} (Chapter~\ref{chapter:discussion}).