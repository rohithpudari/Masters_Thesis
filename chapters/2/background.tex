\startchapter{Background and Related Work}
\label{chapter:background}

\newlength{\savedunitlength}
\setlength{\unitlength}{2em}
\setlength{\unitlength}{\savedunitlength}

\section{Github Copilot}
Copilot is based on OpenAI's Codex models which begins with a GPT-3 model~\cite{Gpt3} and are then fine tuned on code from Github. AlphaCode~\cite{alphacode} from DeepMind is a similar product which tries to generate code suggestions. Due to the restricted access to these models, the quality of the code generated by these models is still unclear and actively researched area including attempts to measure security vulnerabilities~\cite{copilot_security}, correctness~\cite{empirical_eval} and licence infringements~\cite{code_clone}.
In-IDE code completion tools have improved a lot in recent years, from suggesting variables or method calls from user code bases~\cite{mandelin2005} to suggesting entire code blocks~\cite{Ciniselli2021}. 
LLM-based approaches like Copilot and AlphaCode have done remarkably well on improving the state of the art in code completion, solving programming contest style problems~\cite{empirical_eval}. %But it does struggles in complex software engineering problems like avoiding code smells and using best practices. 
% As far as we know, there has not been any research on capabilities of Copilot on complex software engineering challenges (like using idioms and paradigms or design patterns).

Vaithilingam et al.~\cite{Vaithilingam2022} conducted an exploratory study of how developers use Copilot, finding that Copilot did not solve tasks more quickly, but did save time in searching for solutions. More importantly, Copilot seemed to solve the writer's block problem of not knowing how to get started. This notion of seeding an initial, if incorrect, solution is often how design proceeds in software. 

Recent works show initial investigations on how large language models for code can add architecture tactics by using program synthesis~\cite{Shokri2021} and structure learning~\cite{Karmakar2021}.
% Shokri et al.~\cite{Shokri2021} laid out a vision for a program synthesis approach for adding architecture tactics to an existing codebase using the ArCode tool. 
% Karmakar et al.~\cite{Karmakar2021} reported on initial investigations into how language models for code (e.g., CodeBERT), learned structure. Structure learning is likely central to AI approaches to design and architecture. 
Our approach complements these earlier approaches by focusing on moving beyond code completion, where most research effort is currently concentrated.

% and make tools like Copilot suggest the best possible code (by whatever metric you choose--performance, scalability, etc.) for a given scenario.

\section{AI driven Software engineering}
Testing Copilot

\section{Chapter Summary}