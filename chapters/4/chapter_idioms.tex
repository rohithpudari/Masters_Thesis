\startchapter{Language Idioms and Best Practices}
\label{chapter:idioms}

\section{Introduction}
Useful \AISE{} tools should always suggest the optimal way as its first suggestion. In this chapter, We test if \cop{} suggests the optimal way. 
We begin by explaining our approach to \textbf{RQ-1.1} (How do \cct{} manage programming Idioms and Best Practices?). 

To achieve this, We conduct an exploratory study to find how often does current \cct{} suggest the optimal way in their suggestions. In Section~\ref{methodology}, we describe our methodology for the two categories under study: Python Idioms (Section~\ref{secidioms}) and Best Practices in JavaScript (Section~\ref{bp}).
We then compare the suggestions generated by \cop{} and previous works on best practices in Python and JavaScript. We report on how many times \cop{} suggested the optimal way. 

We find that \cop{} performs poorly in suggesting the optimal way in its suggestions. Further, in Section~\ref{evolution}, We provide a detailed discussion on how Best Practices change over time with an example from Asynchronous JavaScript and outline the difficulties of a \cct{} to keep updating its suggestions and reflect the current best practice.

\section{Methodology}
\label{methodology}
In this Section, We explain the methodology we used to address \textbf{RQ-1.1} (How do \cct{} manage programming Idioms and Best Practices?), including how the best practices were sampled (Section~\ref{sampling}), what was the input for copilot (Section~\ref{input}) and how is the suggestion evaluated (Section~\ref{evaluation}). All of the following analysis was carried out using \cop{} extension in Visual Studio Code, which makes \cop{} model versioning and clarity challenging. We use the most recent stable release of Copilot extension (version number todo) in Visual Studio Code.

\subsection{Sampling Approach}
\label{sampling}
To give the best chance for \cop{} to suggest the optimal way, we sampled the top 10 most frequent idioms used in open source projects. So that, \cop{} will have the optimal way more frequently in its training data. However, \cop{} is closed source and we cannot determine if the frequency of code snippet in training data affects \cop{} suggestions in any way. Research by GitHub shows that Copilot can sometimes recite from its training data in ``generic contexts"\footnote{\url{https://github.blog/2021-06-30-github-copilot-research-recitation/}}, which may lead to Licence Infringements shown in Section~\ref{licence}. Sampling the most frequently used Idioms will also help understand if copilot can recite idioms, which is the ideal behaviour for \cct{}.

\subsection{Study Setup}

\subsubsection{Input to \cop{}}
\label{input}
The input to \cop{} consisted of the idiom title as the first comment to provide context, and the input was restricted to being able to derive the ideal way from the input. This is done to ensure \cop{} is making the decision to suggest the good/bad way in its suggestions. This input style also mimics a novice user, who is unaware of the idioms and useful \cct{} should drive the novice user to use best practices.

\subsubsection{Evaluation}
\label{evaluation}
We considered \cop{} suggested the optimal way if Copilot suggested the best practice in the first suggestion, In this case, we considered \cct{} as productivity tools and user should be saving time as opposed to writing the optimal way without using \cct{}, Scrolling through all the suggestions to deduce the optimal way defeats this purpose. For this reason, We restricted ourselves to first suggestion. However, we do note if the best practice appeared any of the top 10 suggestions currently viewable in \cop{} interface. 

% including sections in other files.
\subsection{Paradigms and Idioms}
Paradigms and idioms are our taxonomy's third level of software abstraction. 
This level requires the code suggested by \cct{} to satisfy all the previous levels of abstractions and use common paradigms and language idioms in its code suggestions. 
These include common practices of solving a programming task. 

Returning to our running example of performing a sorting operation on a list of numbers. 
To satisfy this level of abstraction, \cct{} should suggest a syntactically correct list sorting code, using idiomatic ways in its code suggestions, like the Pythonic way of swapping items in a list~(line 5 in figure~\ref{fig:idioms}), As opposed to suggesting non-idiomatic approaches like creating another temporary variable to swap items in a list shown in correctness level~(figure~\ref{fig:correctness}).

Figure~\ref{fig:idioms} shows the sorting example and the Python code suggestions from \cct{} at this abstraction level.

\begin{figure}[hbt!]
    \centering
\begin{tcolorbox}[title=Idioms level suggestion for sort routine,boxsep=.15mm]
    %https://tex.stackexchange.com/questions/337909/tcolorbox-tcbline-style
\textbf{Human Input:}
\begin{lstlisting}[language={Python}]
# sorting a list
arr = [2,4,1,3,7,5]
\end{lstlisting}
\tcbline
\textbf{\cct{} suggestion:}
\begin{lstlisting}[language={Python}]
n = len(arr)
for i in range(n):
	for j in range(n-1):
		if arr[j] > arr[j+1]:
			arr[j], arr[j+1] = arr[j+1], arr[j]
print(arr)
\end{lstlisting}
\end{tcolorbox}
    \caption{Code suggestion of \cct{} at paradigms and idioms level.}
    \label{fig:idioms}
\end{figure}

The goal of this software abstraction level in the taxonomy is for \cct{} to detect and use commonly known idiomatic approaches and paradigms that occur in public code in its suggestions for suggesting code to solve a programming task.

The capabilities required by \cct{} to satisfy paradigms and idioms level of software abstraction are as follows:
\begin{enumerate}
    \item Identify common patterns like paradigms and language idioms in public code repositories~(training data).
    \item Use paradigms and language idioms in suggesting solutions for a programming task.
    \item Satisfy requirements of all the levels below paradigms and idioms in our taxonomy.
\end{enumerate}


\section{Best Practices}
\label{bp}
A good \AISE{} tool should only suggest code that passes code reviews by humans. Code review problems align with the Code Smell level (middle level) in our taxonomy. 

% For example, in JavaScript, callback api was used in the past to achieve concurrency which were replaced by promises. We checked if Copilot suggests code which is specifically mentioned in the JavaScript documentation as a bad practice or an anti-pattern. Bad Practices in using promises for asynchronous JavaScript like not returning promises after creation, forgetting to terminate chains without catch statement, which are explained in documentation\footnote{\url{https://developer.mozilla.org/en-US/docs/Web/JavaScript/Guide/Using_promises}} and StackOverflow\footnote{\url{https://stackoverflow.com/questions/30362733/handling-errors-in-promise-all/}} are not known to Copilot and suggested code with those common anti-patterns as they occur more frequently in Copilot training data.

\noindent\textbf{Method:} We relied on the AirBNB JavaScript coding style guide~\cite{airbnb_code}, a widely used coding style and code review standard. 
The AirBNB standard contains a variety of best practices. 
We chose practices that were closer to the the design level of our taxonomy, rather than the code level (e.g., logging practices rather than trailing comma use in Javascript). 
We focus on five JavaScript coding standards. % from the AirBNB JavaScript coding style guide~\cite{airbnb_code}. 
The input we gave Copilot consisted of the best practice name as the first comment to provide context, and then minimal extra input to prompt Copilot. 
%The listing \textsf{Copying Array Contents} explains our input, Copilot's suggestion, and the ideal/preferred way. 
% This is to ensure Copilot is making the decision to suggest the good/bad way. 
We considered the suggestion a match if Copilot suggested the ideal way as its first suggestion, under the assumption that most developers would not spend time looking through the remaining nine. We do note if the recommended way appeared any of the top 10 suggestions. 
% Below is the list of five best practices we tested:
% \neil{i think a quick list listing the chosen standards would help, alongside the place the suggestion occurred (e.g.).}

Copilot failed to suggest the recommended way for all five standards we tested, i.e, Copilot did not have the recommended way in its top two suggestions. Only 1 out of 5 standards had the recommended approach in the top 10 suggestions. 

Below is an example of Copying Array Contents (Practice 3). 
We first show the user input that triggers Copilot, the top suggestion by Copilot (i.e., model output) and the recommended way suggested by the AirBNB coding style guide~\cite{airbnb_code}.

\begin{tcolorbox}[title=Copying Array Contents,boxsep=.5mm]
    %https://tex.stackexchange.com/questions/337909/tcolorbox-tcbline-style
\textbf{Human Input:}
\begin{lstlisting}[language=JavaScript]
// copying array contents
const len = items.length;
\end{lstlisting}
\tcbline
\textbf{Copilot Suggestion:}
\begin{lstlisting}[language=JavaScript]
const copy = new Array(len);
for (let i = 0; i < len; i++) {
	  copy[i] = items[i];
}
\end{lstlisting}
\tcbline
\textbf{Ideal way\footnote{source \cite{airbnb_code}}:}
\begin{lstlisting}[language=JavaScript]
const itemsCopy = [...items];
\end{lstlisting}
\end{tcolorbox}
\subsection{Evolution of Best Practices}
\label{evolution}
\section{Chapter Summary}
In summary, we start this chapter by showing the methodology used in addressing \textbf{RQ-1.2} (How do \cct{} manage to suggest non-smelly code?). We first introduced the study setup with the input to Copilot and how it was restricted to deriving the best practice from the input and how the suggestions from Copilot were evaluated. We sampled best practices from AirBNB JavaScript coding style guide~\cite{airbnb_code}, and then compared it against Copilot suggestions. Based on results shown in Table~\ref{tab:all_bp}, Copilot struggles to suggest the best practices from widely used coding standards in its suggestions. 

In this chapter, we showed that Copilot struggles to detect and follow coding style guides  present in public repositories of GitHub and always suggest code that follows those coding style guides. The ideal behavior of \cct{} like Copilot in solving this problem is to detect the coding style guideline from existing code in the project and always suggest code that follows the guideline.

In the next chapter (chapter~\ref{chapter:framework}), we illustrate our taxonomy inspired from autonomous driving levels on the software abstraction hierarchy in \AISE{}, and delineate where \cct{} are currently best able to perform, and where more complex software engineering tasks overwhelm them.