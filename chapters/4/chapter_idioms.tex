\startchapter{Language Idioms and Best Practices}
\label{chapter:idioms}

\section{Introduction}
Useful \AISE{} tools should only suggest idiomatic code, the Paradigms and Idioms level in our taxonomy.  
We leveraged the work of Alexandru et al.~\cite{Alexandru2018}, which identified community consensus on common Python programming idioms, and examined how Copilot suggested their top five idioms.


\section{Methodology}
The input to Copilot consisted of the idiom title as the first comment to provide context, as before, with minimal additional prompting. This mimics a novice user. 
% and the input was restricted to being able to go derive the ideal way from the input. This is to ensure Copilot is making the decision to suggest the good/bad way.
% We considered it a match if Copilot suggested the idiom in the first two suggestions, but note if the idiom appeared any of the top 10 suggestions currently viewable in Copilot. 
% \neil{is there evidence in recommender systems HCI to support top2?}\rohith{we could navigate through all suggestions with "alt+]" and "alt+[" listed in \href{https://github.com/github/copilot-docs/blob/main/docs/visualstudiocode/gettingstarted.md}{click here}}

% including sections in other files.
\section{Language Idioms and Paradigms}
We leveraged the work of Alexandru et al.~\cite{Alexandru2018}, which identified commmunity consensus on common Python programming idioms, and compared how the top five idioms from their work were suggested by Copilot.

\noindent\textbf{Method:} %\neil{need to explain how the idiom was triggered - what would make Copilot suggest it?}
The input of copilot consisted of the idiom title as the first comment to provide context and the input was restricted to being able to go derive the ideal way from the input. This is to ensure copilot is making the decision to suggest the good/bad way.
We considered it a match if Copilot suggested the idiom in the first two suggestions, but note if the idiom appeared any of the top 10 suggestions currently viewable in Copilot. 
\neil{is there evidence in recommender systems HCI to support top2?}\rohith{we could navigate through all suggestions with "alt+]" and "alt+[" listed in \href{https://github.com/github/copilot-docs/blob/main/docs/visualstudiocode/gettingstarted.md}{click here}}
Below is the list of five idioms we tested:
\begin{itemize}
    \item Idiom 1: List Comprehension (no suggestion matched)
    \item Idiom 2: Dict Comprehension (no suggestion matched)
    \item Idiom 3: Mapping (place: 9th)
    \item Idiom 4: Filter (place: 7th)
    \item Idiom 5: Reduce (place: 9th)
\end{itemize}

Copilot failed to suggest the idiomatic approach for all five idioms we tested, i.e, Copilot did not have the recommended solution in its top 2 suggestions. However, for 3 out of 5 idioms, the idiomatic way was in Copilot's top 10 suggestions.

Below is the example of list comprehension idiom (Idiom 1), showing the input, the top suggestion by Copilot(i.e. model output) and the recommended way suggested by Alexandru et al.~\cite{Alexandru2018}.

\begin{tcolorbox}[title=List Comprehension,boxsep=.5mm]
    %https://tex.stackexchange.com/questions/337909/tcolorbox-tcbline-style
\textbf{Human Input:}
\begin{lstlisting}[language=Python]
#list comprehension

result_list = 
\end{lstlisting}
\tcbline
\textbf{Copilot Suggestion:}
\begin{lstlisting}[language=Python,escapechar=\%]
% \noindent\textcolor{gray}{result\_list  =} % []
for i in range(1,11):
    result_list.append(i)
\end{lstlisting}
\tcbline
\textbf{Ideal way\footnote{source \cite{Alexandru2018}}:}
\begin{lstlisting}[language=Python]
result_list = [el for el in range(11)]
\end{lstlisting}
\end{tcolorbox}

The ranking metric for different suggestions made by Copilot has not been made public. Thus, we cannot determine the reason Copilot is ranking one approach (non-idiomatic) over the idiomatic (preferred) approach. However, large language model suggestions are based on its training data~\cite{training_extraction}, so one explanation is that the non-idiomatic approach is more frequent in the training data~\cite{stochastic_parrots}. We comment on explainability later. 

% \neil{rohith: verify this LLM approach is the case. We might want to cite the Codex paper}, 
% \rohith{They never disclosed how the ranking system works, but there are some other papers discussing about it, added them above}

\section{Best Practices}
\label{bp}
A good \AISE{} tool should only suggest code that passes code reviews by humans. Code review problems align with the Code Smell level (middle level) in our taxonomy. 

% For example, in JavaScript, callback api was used in the past to achieve concurrency which were replaced by promises. We checked if Copilot suggests code which is specifically mentioned in the JavaScript documentation as a bad practice or an anti-pattern. Bad Practices in using promises for asynchronous JavaScript like not returning promises after creation, forgetting to terminate chains without catch statement, which are explained in documentation\footnote{\url{https://developer.mozilla.org/en-US/docs/Web/JavaScript/Guide/Using_promises}} and StackOverflow\footnote{\url{https://stackoverflow.com/questions/30362733/handling-errors-in-promise-all/}} are not known to Copilot and suggested code with those common anti-patterns as they occur more frequently in Copilot training data.

\noindent\textbf{Method:} We relied on the AirBNB JavaScript coding style guide~\cite{airbnb_code}, a widely used coding style and code review standard. 
The AirBNB standard contains a variety of best practices. 
We chose practices that were closer to the the design level of our taxonomy, rather than the code level (e.g., logging practices rather than trailing comma use in Javascript). 
We focus on five JavaScript coding standards. % from the AirBNB JavaScript coding style guide~\cite{airbnb_code}. 
The input we gave Copilot consisted of the best practice name as the first comment to provide context, and then minimal extra input to prompt Copilot. 
%The listing \textsf{Copying Array Contents} explains our input, Copilot's suggestion, and the ideal/preferred way. 
% This is to ensure Copilot is making the decision to suggest the good/bad way. 
We considered the suggestion a match if Copilot suggested the ideal way as its first suggestion, under the assumption that most developers would not spend time looking through the remaining nine. We do note if the recommended way appeared any of the top 10 suggestions. 
% Below is the list of five best practices we tested:
% \neil{i think a quick list listing the chosen standards would help, alongside the place the suggestion occurred (e.g.).}

Copilot failed to suggest the recommended way for all five standards we tested, i.e, Copilot did not have the recommended way in its top two suggestions. Only 1 out of 5 standards had the recommended approach in the top 10 suggestions. 

Below is an example of Copying Array Contents (Practice 3). 
We first show the user input that triggers Copilot, the top suggestion by Copilot (i.e., model output) and the recommended way suggested by the AirBNB coding style guide~\cite{airbnb_code}.

\begin{tcolorbox}[title=Copying Array Contents,boxsep=.5mm]
    %https://tex.stackexchange.com/questions/337909/tcolorbox-tcbline-style
\textbf{Human Input:}
\begin{lstlisting}[language=JavaScript]
// copying array contents
const len = items.length;
\end{lstlisting}
\tcbline
\textbf{Copilot Suggestion:}
\begin{lstlisting}[language=JavaScript]
const copy = new Array(len);
for (let i = 0; i < len; i++) {
	  copy[i] = items[i];
}
\end{lstlisting}
\tcbline
\textbf{Ideal way\footnote{source \cite{airbnb_code}}:}
\begin{lstlisting}[language=JavaScript]
const itemsCopy = [...items];
\end{lstlisting}
\end{tcolorbox}
\subsection{Evolution of Best Practices}
\label{evolution}
\section{Chapter Summary}

In summary, we start this chapter by showing the methodology used in addressing \textbf{RQ-1.2} (How do \cct{} manage programming Idioms and Best Practices?). We first introduced the study setup with how the input to \cop{} was restricted and how the suggestion from \cop{} were evaluated. We sampled best practices in Python and Javascript and then compared it against \cop{} suggestions. 

Based on results shown in Table~\ref{tab:all_idioms} and \ref{tab:all_bp}, \cop{} struggles to suggest the best practices in its suggestions. In the next chapter (chapter~\ref{chapter:framework}), we illustrate our taxonomy on the hierarchy in \AISE{}, and delineate where \cct{} are currently best able to perform, and where more complex software engineering tasks overwhelm them.