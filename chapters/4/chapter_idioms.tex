\startchapter{Language Idioms and Best Practices}
\label{chapter:idioms}

\section{Introduction}
Useful \AISE{} tools should only suggest idiomatic code, the Paradigms and Idioms level in our taxonomy.  
We leveraged the work of Alexandru et al.~\cite{Alexandru2018}, which identified community consensus on common Python programming idioms, and examined how Copilot suggested their top five idioms.


\section{Methodology}
The input to Copilot consisted of the idiom title as the first comment to provide context, as before, with minimal additional prompting. This mimics a novice user. 
% and the input was restricted to being able to go derive the ideal way from the input. This is to ensure Copilot is making the decision to suggest the good/bad way.
% We considered it a match if Copilot suggested the idiom in the first two suggestions, but note if the idiom appeared any of the top 10 suggestions currently viewable in Copilot. 
% \neil{is there evidence in recommender systems HCI to support top2?}\rohith{we could navigate through all suggestions with "alt+]" and "alt+[" listed in \href{https://github.com/github/copilot-docs/blob/main/docs/visualstudiocode/gettingstarted.md}{click here}}

% including sections in other files.
\subsection{Paradigms and Idioms}
Paradigms and idioms are our taxonomy's third level of software abstraction. 
This level requires the code suggested by \cct{} to satisfy all the previous levels of abstractions and use common paradigms and language idioms in its code suggestions. 
These include common practices of solving a programming task. 

Returning to our running example of performing a sorting operation on a list of numbers. 
To satisfy this level of abstraction, \cct{} should suggest a syntactically correct list sorting code, using idiomatic ways in its code suggestions, like the Pythonic way of swapping items in a list~(line 5 in figure~\ref{fig:idioms}), As opposed to suggesting non-idiomatic approaches like creating another temporary variable to swap items in a list shown in correctness level~(figure~\ref{fig:correctness}).

Figure~\ref{fig:idioms} shows the sorting example and the Python code suggestions from \cct{} at this abstraction level.

\begin{figure}[hbt!]
    \centering
\begin{tcolorbox}[title=Idioms level suggestion for sort routine,boxsep=.15mm]
    %https://tex.stackexchange.com/questions/337909/tcolorbox-tcbline-style
\textbf{Human Input:}
\begin{lstlisting}[language={Python}]
# sorting a list
arr = [2,4,1,3,7,5]
\end{lstlisting}
\tcbline
\textbf{\cct{} suggestion:}
\begin{lstlisting}[language={Python}]
n = len(arr)
for i in range(n):
	for j in range(n-1):
		if arr[j] > arr[j+1]:
			arr[j], arr[j+1] = arr[j+1], arr[j]
print(arr)
\end{lstlisting}
\end{tcolorbox}
    \caption{Code suggestion of \cct{} at paradigms and idioms level.}
    \label{fig:idioms}
\end{figure}

The goal of this software abstraction level in the taxonomy is for \cct{} to detect and use commonly known idiomatic approaches and paradigms that occur in public code in its suggestions for suggesting code to solve a programming task.

The capabilities required by \cct{} to satisfy paradigms and idioms level of software abstraction are as follows:
\begin{enumerate}
    \item Identify common patterns like paradigms and language idioms in public code repositories~(training data).
    \item Use paradigms and language idioms in suggesting solutions for a programming task.
    \item Satisfy requirements of all the levels below paradigms and idioms in our taxonomy.
\end{enumerate}


\section{Best Practices}
\label{bp}
A good \AISE{} tool should only suggest code that passes code reviews by humans. Code review problems align with the Code Smell level (middle level) in our taxonomy. 

% For example, in JavaScript, callback api was used in the past to achieve concurrency which were replaced by promises. We checked if Copilot suggests code which is specifically mentioned in the JavaScript documentation as a bad practice or an anti-pattern. Bad Practices in using promises for asynchronous JavaScript like not returning promises after creation, forgetting to terminate chains without catch statement, which are explained in documentation\footnote{\url{https://developer.mozilla.org/en-US/docs/Web/JavaScript/Guide/Using_promises}} and StackOverflow\footnote{\url{https://stackoverflow.com/questions/30362733/handling-errors-in-promise-all/}} are not known to Copilot and suggested code with those common anti-patterns as they occur more frequently in Copilot training data.

\noindent\textbf{Method:} We relied on the AirBNB JavaScript coding style guide~\cite{airbnb_code}, a widely used coding style and code review standard. 
The AirBNB standard contains a variety of best practices. 
We chose practices that were closer to the the design level of our taxonomy, rather than the code level (e.g., logging practices rather than trailing comma use in Javascript). 
We focus on five JavaScript coding standards. % from the AirBNB JavaScript coding style guide~\cite{airbnb_code}. 
The input we gave Copilot consisted of the best practice name as the first comment to provide context, and then minimal extra input to prompt Copilot. 
%The listing \textsf{Copying Array Contents} explains our input, Copilot's suggestion, and the ideal/preferred way. 
% This is to ensure Copilot is making the decision to suggest the good/bad way. 
We considered the suggestion a match if Copilot suggested the ideal way as its first suggestion, under the assumption that most developers would not spend time looking through the remaining nine. We do note if the recommended way appeared any of the top 10 suggestions. 
% Below is the list of five best practices we tested:
% \neil{i think a quick list listing the chosen standards would help, alongside the place the suggestion occurred (e.g.).}

Copilot failed to suggest the recommended way for all five standards we tested, i.e, Copilot did not have the recommended way in its top two suggestions. Only 1 out of 5 standards had the recommended approach in the top 10 suggestions. 

Below is an example of Copying Array Contents (Practice 3). 
We first show the user input that triggers Copilot, the top suggestion by Copilot (i.e., model output) and the recommended way suggested by the AirBNB coding style guide~\cite{airbnb_code}.

\begin{tcolorbox}[title=Copying Array Contents,boxsep=.5mm]
    %https://tex.stackexchange.com/questions/337909/tcolorbox-tcbline-style
\textbf{Human Input:}
\begin{lstlisting}[language=JavaScript]
// copying array contents
const len = items.length;
\end{lstlisting}
\tcbline
\textbf{Copilot Suggestion:}
\begin{lstlisting}[language=JavaScript]
const copy = new Array(len);
for (let i = 0; i < len; i++) {
	  copy[i] = items[i];
}
\end{lstlisting}
\tcbline
\textbf{Ideal way\footnote{source \cite{airbnb_code}}:}
\begin{lstlisting}[language=JavaScript]
const itemsCopy = [...items];
\end{lstlisting}
\end{tcolorbox}
\subsection{Evolution of Best Practices}
\label{evolution}
\section{Chapter Summary}
In summary, we start this chapter by showing the methodology used in addressing \textbf{RQ-1.2} (How do \cct{} manage to suggest non-smelly code?). We first introduced the study setup with the input to Copilot and how it was restricted to deriving the best practice from the input and how the suggestions from Copilot were evaluated. We sampled best practices from AirBNB JavaScript coding style guide~\cite{airbnb_code}, and then compared it against Copilot suggestions. Based on results shown in Table~\ref{tab:all_bp}, Copilot struggles to suggest the best practices from widely used coding standards in its suggestions. 

In this chapter, we showed that Copilot struggles to detect and follow coding style guides  present in public repositories of GitHub and always suggest code that follows those coding style guides. The ideal behavior of \cct{} like Copilot in solving this problem is to detect the coding style guideline from existing code in the project and always suggest code that follows the guideline.

In the next chapter (chapter~\ref{chapter:framework}), we illustrate our taxonomy inspired from autonomous driving levels on the software abstraction hierarchy in \AISE{}, and delineate where \cct{} are currently best able to perform, and where more complex software engineering tasks overwhelm them.