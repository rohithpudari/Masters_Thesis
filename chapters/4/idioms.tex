\subsection{Paradigms and Idioms}
Paradigms and idioms are our taxonomy's third level of software abstraction. 
This level requires the code suggested by \cct{} to satisfy all the previous levels of abstractions and use common paradigms and language idioms in its code suggestions. 
These include common practices of solving a programming task. 

Returning to our running example of performing a sorting operation on a list of numbers. 
To satisfy this level of abstraction, \cct{} should suggest a syntactically correct list sorting code, using idiomatic ways in its code suggestions, like the Pythonic way of swapping items in a list~(line 5 in figure~\ref{fig:idioms}), As opposed to suggesting non-idiomatic approaches like creating another temporary variable to swap items in a list shown in correctness level~(figure~\ref{fig:correctness}).

Figure~\ref{fig:idioms} shows the sorting example and the Python code suggestions from \cct{} at this abstraction level.

\begin{figure}[hbt!]
    \centering
\begin{tcolorbox}[title=Idioms level suggestion for sort routine,boxsep=.15mm]
    %https://tex.stackexchange.com/questions/337909/tcolorbox-tcbline-style
\textbf{Human Input:}
\begin{lstlisting}[language={Python}]
# sorting a list
arr = [2,4,1,3,7,5]
\end{lstlisting}
\tcbline
\textbf{\cct{} suggestion:}
\begin{lstlisting}[language={Python}]
n = len(arr)
for i in range(n):
	for j in range(n-1):
		if arr[j] > arr[j+1]:
			arr[j], arr[j+1] = arr[j+1], arr[j]
print(arr)
\end{lstlisting}
\end{tcolorbox}
    \caption{Code suggestion of \cct{} at paradigms and idioms level.}
    \label{fig:idioms}
\end{figure}

The goal of this software abstraction level in the taxonomy is for \cct{} to detect and use commonly known idiomatic approaches and paradigms that occur in public code in its suggestions for suggesting code to solve a programming task.

The capabilities required by \cct{} to satisfy paradigms and idioms level of software abstraction are as follows:
\begin{enumerate}
    \item Identify common patterns like paradigms and language idioms in public code repositories~(training data).
    \item Use paradigms and language idioms in suggesting solutions for a programming task.
    \item Satisfy requirements of all the levels below paradigms and idioms in our taxonomy.
\end{enumerate}

