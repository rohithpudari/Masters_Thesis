\section{Python Idioms}
\label{secidioms}
We sampled idioms from the work of Alexandru et al.~\cite{Alexandru2018}, which identified idioms from various popular open source projects and books.
We leveraged the work of Alexandru et al.~\cite{Alexandru2018}, which identified community consensus on common Python programming idioms, and examined how Copilot suggested their top five idioms.

Copilot failed to suggest the idiomatic approach for all five idioms we tested. 
For 3 out of 5 idioms, the idiomatic way was in Copilot's top 10 suggestions.

Below is the example of the List Comprehension idiom, showing user input, the top suggestion by Copilot (i.e., model output) and the recommended way suggested by Alexandru et al.~\cite{Alexandru2018}.

\begin{tcolorbox}[title=List Comprehension,boxsep=.5mm]
    %https://tex.stackexchange.com/questions/337909/tcolorbox-tcbline-style
\textbf{Human Input:}
\begin{lstlisting}[language={Python}]
#list comprehension
result_list = 
\end{lstlisting}
\tcbline
\textbf{Copilot Suggestion:}
\begin{lstlisting}[language=Python,escapechar=\%]
% \noindent\textcolor{gray}{result\_list  =} % []
for i in range(1,11):
    result_list.append(i)
\end{lstlisting}
\tcbline
\textbf{Ideal way\footnote{source \cite{Alexandru2018}}:}
\begin{lstlisting}[language=Python]
result_list = [el for el in range(11)]
\end{lstlisting}
\end{tcolorbox}
