\startchapter{AI-driven Software Engineering}
\label{chapter:Exp}

The current approaches focus on programming-in-the-small \cite{DeRemer1976} i.e, on individual lines of code. 
Code language models have focused (effectively!) on source code as natural language~\cite{natural}.
This models the software development task as predicting the next token or series of tokens.
As Copilot's webpage says, the aim for Copilot is to produce ``safe and effective code [with] suggestions for whole lines or entire functions'' as ``your AI pair programmer''~\cite{Copilot-web}. 
Thus support from language models is currently focused on software \textit{programming}, rather than software \emph{engineering} (in-the-large). 
% \neil{wut? --D} 
% But there are areas of code suggestions that are not easily fixable or at least, not detectable efficiently before training or recommendation is required, at the time-scale of code completion (which we assume is nearly instantaneous, as evidenced by how existing autocomplete editors work.)

What might AI support for these more complex software engineering challenges require? 
We present a simple taxonomy in Figure \ref{fig:levels}, modeled on the ideas in the SAE taxonomy and Koopman's extension for self-driving vehicles (Fig. \ref{fig:koopman_pyramid}), as one way of thinking about this question. 

\section{Taxonomy}
Our taxonomy is a software abstraction hierarchy where `basic programming functionality' such as code compilation and syntax checking is at the least abstract level.
Software architecture analysis and design is at the most abstract level.
As we ascend the levels, just as with Koopman's pyramid in Figure \ref{fig:koopman_pyramid}, software challenges a) rely on successfully solving the challenges below, and b) become more difficult to automate (e.g., crafting design rules vs code smells). 

The challenges further up the hierarchy are nonetheless more important for software quality attributes (QA) \cite{Ernst2017} and for a well-engineered software system.
For example, an automated solution to the top level of the taxonomy would be able to follow heuristics to engineer a well designed software system, one which would be easy to modify and scale to sudden changes in use.

The question remains where we stand today, and where the line (or area) exists between places where the AI can take control, and where a human engineer should be involved, just like the distinction between a self-driving car and one where humans still need to be alert.
Our taxonomy in Figure \ref{fig:levels} captures this as a distinction between what seems automatable or is already automated (green, lower) and levels that are not yet automated (red, higher).
% Syntax reflects syntactically correct code and is detected by the compiler. Standards and warnings include `-w' style compiler flags. Smells are aspects of source code that are sub-optimal by general acceptance; paradigms and idioms are good practices for languages or application styles. Finally, design refers to optimal design approaches for system quality attributes.

\begin{figure}
    \includegraphics[width=.7\linewidth]{Figures/taxonomy-copilot.pdf}
    \caption{Hierarchy of software abstractions. Levels depend on one another as abstraction increases. Parentheticals are examples of each level. As we move up the hierarchy, we require more human oversight of the AI; as we move down the hierarchy, rules for detecting problems are easier to formulate. Green levels are areas where AI supported programming works reasonably well, while red levels are difficult or impossible, as we show in the paper. }
    \label{fig:levels}
\end{figure}

To motivate this taxonomy, we look at how Copilot, as a representative AI supported programming tool, is able to produce code that aligns with two of the levels in our taxonomy. 
One is the way Copilot manages \emph{language idioms}.
A language idiom~\cite{Alexandru2018} is commonly thought of as the `one way to do it', where `it' might include looping constructs, storing data, or initializing data structures.
The second is the way Copilot is able to \emph{prevent code smells} by suggesting and understanding community-agreed best practices in programming, for example, in conducting code reviews.
% \neil{third? maybe SQ smells design smells.}.
For each level, we compare the recommended \textbf{ideal} solution with the solution Copilot suggests. Since Copilot generates up to 10 suggestions, we also consider where the suggestion appears in the list (if at all). 
\section{Language Idioms and Paradigms}
We leveraged the work of Alexandru et al.~\cite{Alexandru2018}, which identified commmunity consensus on common Python programming idioms, and compared how the top five idioms from their work were suggested by Copilot.

\noindent\textbf{Method:} %\neil{need to explain how the idiom was triggered - what would make Copilot suggest it?}
The input of copilot consisted of the idiom title as the first comment to provide context and the input was restricted to being able to go derive the ideal way from the input. This is to ensure copilot is making the decision to suggest the good/bad way.
We considered it a match if Copilot suggested the idiom in the first two suggestions, but note if the idiom appeared any of the top 10 suggestions currently viewable in Copilot. 
\neil{is there evidence in recommender systems HCI to support top2?}\rohith{we could navigate through all suggestions with "alt+]" and "alt+[" listed in \href{https://github.com/github/copilot-docs/blob/main/docs/visualstudiocode/gettingstarted.md}{click here}}
Below is the list of five idioms we tested:
\begin{itemize}
    \item Idiom 1: List Comprehension (no suggestion matched)
    \item Idiom 2: Dict Comprehension (no suggestion matched)
    \item Idiom 3: Mapping (place: 9th)
    \item Idiom 4: Filter (place: 7th)
    \item Idiom 5: Reduce (place: 9th)
\end{itemize}

Copilot failed to suggest the idiomatic approach for all five idioms we tested, i.e, Copilot did not have the recommended solution in its top 2 suggestions. However, for 3 out of 5 idioms, the idiomatic way was in Copilot's top 10 suggestions.

Below is the example of list comprehension idiom (Idiom 1), showing the input, the top suggestion by Copilot(i.e. model output) and the recommended way suggested by Alexandru et al.~\cite{Alexandru2018}.

\begin{tcolorbox}[title=List Comprehension,boxsep=.5mm]
    %https://tex.stackexchange.com/questions/337909/tcolorbox-tcbline-style
\textbf{Human Input:}
\begin{lstlisting}[language=Python]
#list comprehension

result_list = 
\end{lstlisting}
\tcbline
\textbf{Copilot Suggestion:}
\begin{lstlisting}[language=Python,escapechar=\%]
% \noindent\textcolor{gray}{result\_list  =} % []
for i in range(1,11):
    result_list.append(i)
\end{lstlisting}
\tcbline
\textbf{Ideal way\footnote{source \cite{Alexandru2018}}:}
\begin{lstlisting}[language=Python]
result_list = [el for el in range(11)]
\end{lstlisting}
\end{tcolorbox}

The ranking metric for different suggestions made by Copilot has not been made public. Thus, we cannot determine the reason Copilot is ranking one approach (non-idiomatic) over the idiomatic (preferred) approach. However, large language model suggestions are based on its training data~\cite{training_extraction}, so one explanation is that the non-idiomatic approach is more frequent in the training data~\cite{stochastic_parrots}. We comment on explainability later. 

% \neil{rohith: verify this LLM approach is the case. We might want to cite the Codex paper}, 
% \rohith{They never disclosed how the ranking system works, but there are some other papers discussing about it, added them above}

\section{Code Review Best Practices}
We considered best practices for code review in JavaScript development. 
This aligns with the Code Smell level (middle tier) in our taxonomy. 
To ground our practices we relied on the AirBNB JavaScript coding style guide~\cite{airbnb_code}, a widely used coding style and code review standard. 
The AirBNB standard contains a variety of best practices. 
To scope our approach, we chose practices that were closer to the the design level of our taxonomy, rather than the code level (e.g., trailing comma use in Javascript).

% For example, in JavaScript, callback api was used in the past to achieve concurrency which were replaced by promises. We checked if copilot suggests code which is specifically mentioned in the JavaScript documentation as a bad practice or an anti-pattern. Bad Practices in using promises for asynchronous JavaScript like not returning promises after creation, forgetting to terminate chains without catch statement, which are explained in documentation\footnote{\url{https://developer.mozilla.org/en-US/docs/Web/JavaScript/Guide/Using_promises}} and StackOverflow\footnote{\url{https://stackoverflow.com/questions/30362733/handling-errors-in-promise-all/}} are not known to copilot and suggested code with those common anti-patterns as they occur more frequently in copilot training data.

\noindent\textbf{Method:} We chose 5 JavaScript coding standards from the AirBNB JavaScript coding style guide~\cite{airbnb_code}. We considered it a match if Copilot suggested the recommended way in the first 2 suggestions, but note if the recommended way appeared any of the top 10 suggestions. 
Below is the list of five best practices we tested:
% \neil{i think a quick list listing the chosen standards would help, alongside the place the suggestion occurred (e.g.).}

\begin{itemize}
    \item Practice 1: Object Shorthand: Usage of object method shorthand (no suggestion matched)
    \item Practice 2: Array Creating Constructor (place: 6th)
    \item Practice 3: Copying Array Contents (no suggestion matched)
    \item Practice 4: Logging a Function (no suggestion matched)
    \item Practice 5: Exporting a Function (no suggestion matched)
\end{itemize}

Copilot failed to suggest the recommended way for all five standards we tested, i.e, Copilot did not have the recommended way in its top two suggestions. Only 1 out of 5 standards had the recommended way in the top 10 suggestions. 

Below is an example of logging a function (Practice 4). 
We first show the user input that triggers Copilot, the top suggestion by Copilot (i.e. model output) and the recommended way suggested by the AirBNB coding style guide~\cite{airbnb_code}.

\begin{tcolorbox}[title=Logging a Function,boxsep=.5mm]
    %https://tex.stackexchange.com/questions/337909/tcolorbox-tcbline-style
\textbf{Human Input:}
\begin{lstlisting}[language=JavaScript]
// logging function

class foo {
    bar ()
\end{lstlisting}
\tcbline
\textbf{Copilot Suggestion:}
\begin{lstlisting}[language=JavaScript]
{
    console.log('bar');
    }
}
\end{lstlisting}
\tcbline
\textbf{Ideal way\footnote{source~\cite{airbnb_code}}:}
\begin{lstlisting}[language=JavaScript]
class foo {
    bar (){ 
    console.log(this.bar) ;
}}
\end{lstlisting}
\end{tcolorbox}

\subsection{Code Smells}
The code smells level is the penultimate level of software abstractions in our taxonomy. 
This level requires the suggested code satisfy all the previous levels of abstractions and avoid common code smells in its suggestions, these include common bad practices found in public code. This software abstraction level is also requires \cct{} to suggest the most optimized version of all its possible code suggestions.

For example, considering the task of performing a sorting operation on a list of numbers. To satisfy this level of abstraction, \cct{} should suggest a syntactically correct list sorting code, using common patterns like idioms but not including common code smells that occur in public code like resolving edge cases. 
Figure~\ref{fig:smells} shows the example and the suggestion from \cct{} at this abstraction level.

\begin{figure}[hbt!]
    \centering
    \includegraphics[width=\linewidth]{Figures/smells.png}
    \caption{\cct{} code smells level suggestions}
    \label{fig:smells}
\end{figure}

The main goals of this level of software abstraction in our taxonomy is for \cct{} to be able to detect and avoid bad practices such as code smells that commonly occur in public code in its code suggestions to a problem and to suggest the most optimized version as its first suggestion to a programming task.

The capabilities required by \cct{} to satisfy this level of abstraction are as follows:
\begin{enumerate}
    \item Identify common bad practices such as code smells that occur in public code~(training data).
    \item Suggest solutions that does not have any code smells.
    \item Suggested code should the most optimized version of all the possible suggestions \cct{} could create for a given problem.
    \item Satisfy requirements of all the levels below code smells in our taxonomy.
\end{enumerate}

% \begin{tcolorbox}[title=Code smells level suggestion for sort routine,boxsep=.15mm]
%     %https://tex.stackexchange.com/questions/337909/tcolorbox-tcbline-style
% \textbf{Human Input:}
% \begin{lstlisting}[language={Python}]
% # sorting a list
% arr = [2,4,1,3,7,5]
% \end{lstlisting}
% \tcbline
% \textbf{\cct{} suggestion:}
% \begin{lstlisting}[language={Python}, morekeywords={False, True}]
% swapped = True
% n = len(arr)
% j = 0
% while(swapped==True and n>1):
% 	swapped = False
% 	for i in range(n-j-1):
% 	    if arr[i] > arr[i+1]:
% 	        arr[i], arr[i+1] = arr[i+1], arr[i]
% 	        swapped = True
% 	n-=1
% 	j+=1
% print(arr)
% \end{lstlisting}
% \end{tcolorbox}
\section{Chapter Summary}

In summary, we start this chapter by showing the methodology used in addressing \textbf{RQ-1.2} (How do \cct{} manage programming Idioms and Best Practices?). We first introduced the study setup with how the input to \cop{} was restricted and how the suggestion from \cop{} were evaluated. We sampled best practices in Python and Javascript and then compared it against \cop{} suggestions. 

Based on results shown in Table~\ref{tab:all_idioms} and \ref{tab:all_bp}, \cop{} struggles to suggest the best practices in its suggestions. In the next chapter (chapter~\ref{chapter:framework}), we illustrate our taxonomy on the hierarchy in \AISE{}, and delineate where \cct{} are currently best able to perform, and where more complex software engineering tasks overwhelm them.