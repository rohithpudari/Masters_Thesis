\section{Chapter Summary}
In this chapter, we introduced the taxonomy of software abstraction hierarchy. We explain each software abstraction level in the taxonomy and the capabilities required by AI-supported code completion tools to satisfy each software abstraction level.
Additionally our analysis of Copilot performance in suggesting Pythonic idioms and best practices in JavaScript led us to create a metric for all \cct{} to help answer \textbf{RQ-1} (What are the current boundaries of code completion tools). We found that Copilot satisfies syntax and correctness levels in our taxonomy.

Finally, we addressed \textbf{RQ-2} (Given the current boundary, how far is it from suggesting design decisions?). We began with a discussion on the existing capabilities of Copilot and the desired capabilities of \cct{} at the design abstraction level. We then observed that software design evolves over time and \cct{} needs to be updated often to keep its code suggestions up-to-date.

Identifying the limitations of current \cct{} like Copilot could help in resolving the issues of found on Copilot and also help in the development of new tools that can satisfy design level of our taxonomy and help users make better software. We will discuss this and other potential solutions in more detail in the next chapter~(Chapter~\ref{chapter:discussion}).