\section{Chapter Summary}
In summary, we start this chapter by showing the methodology used in addressing \textbf{RQ-1.2} (How do \cct{} manage to suggest non-smelly code?). We first introduced the study setup with the input to Copilot and how it was restricted to deriving the best practice from the input and how the suggestions from Copilot were evaluated. We sampled best practices from AirBNB JavaScript coding style guide~\cite{airbnb_code}, and then compared it against Copilot suggestions. Based on results shown in Table~\ref{tab:all_bp}, Copilot struggles to suggest the best practices from widely used coding standards in its suggestions. 

In this chapter, we showed that Copilot struggles to detect and follow coding style guides  present in public repositories of GitHub and always suggest code that follows those coding style guides. The ideal behavior of \cct{} like Copilot in solving this problem is to detect the coding style guideline from existing code in the project and always suggest code that follows the guideline.

In the next chapter (chapter~\ref{chapter:framework}), we illustrate our taxonomy inspired from autonomous driving levels on the software abstraction hierarchy in \AISE{}, and delineate where \cct{} are currently best able to perform, and where more complex software engineering tasks overwhelm them.