\subsection{Syntax}
\label{syntax}
The syntax level is the lowest software abstraction level in our taxonomy. 
This level includes the most basic programming functionality like suggesting code that has correct syntax and has no errors in code compilations. 
This level does not require the \cct{} suggested code to successfully solve the programming task but to suggest code without any obvious errors like syntax or code compilation errors.

For example, consider a programming task of performing a sorting operation on a list of numbers. 
To satisfy this level of abstraction, \cct{} should suggest code that is syntactically correct without any compilation errors and the code is not required to perform the sorting operation correctly. 
Figure~\ref{fig:syntax} shows the sorting example and Python syntax suggestions from \cct{} at this abstraction level.

\begin{figure}[hbt!]
    \centering
\begin{tcolorbox}[title=Syntax level suggestion for sort routine,boxsep=.15mm]
    %https://tex.stackexchange.com/questions/337909/tcolorbox-tcbline-style
\textbf{Human Input:}
\begin{lstlisting}[language={Python}]
# sorting a list
arr = [2,4,1,3,7,5]
\end{lstlisting}
\tcbline
\textbf{\cct{} suggestion:}
\begin{lstlisting}[language={Python}]
for i in range( ):
\end{lstlisting}
\end{tcolorbox}
    \caption{Code suggestion of \cct{} at syntax level.}
    \label{fig:syntax}
\end{figure}

The goal of this software abstraction level in our taxonomy is for a \cct{} to be able to suggest code without any syntactical errors.
The capabilities required by \cct{} to satisfy this level of abstraction are as follows:

\begin{enumerate}
    \item Suggested code should be syntactically correct.
    \item Suggested code should not produce any errors in code compilation.
\end{enumerate}

